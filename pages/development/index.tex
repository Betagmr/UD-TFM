\section{Desarrollo del Proyecto}
En esta sección se detallan los aspectos más relevantes del desarrollo del proyecto.
Se profundizará en los diferentes aspectos del diseño, la implementación y la
prueba de los sistemas y componentes desarrollados. Además, se describirán las
herramientas y tecnologías utilizadas, así como los procesos y metodologías
empleadas para el desarrollo del proyecto. A continuación, se muestra una vista general de los diferentes elementos
que conforman la estructura del marco de trabajo. la Figura \ref{fig:general-vision} muestra como las diferentes
partes del proyectos se integran con los nuevos desarrollos que se realizan dentro de la empresa.

\begin{figure}[ht]
    \centering
    \includegraphics[width=\textwidth]{general-vision-2.png}
    \caption{Vista general del proyecto}
    \label{fig:general-vision}
\end{figure}

Dentro de la estructura general del proyecto, se pueden identificar varios elementos
que conforman el proyecto: la infraestructura y herramientas, el catálogo de componentes y 
plantillas, la documentación del proyecto y el repositorio de artefactos. Cada uno de estos elementos 
se encarga de aspectos diferentes dentro del ecosistema del proyecto, y se relacionan entre sí para formar
un sistema completo. \medskip

La idea principal es que la infraestructura sea complementaria
al desarrollo, estando a la disposición de los investigadores de una forma sencilla. El catálogo
de componentes y plantillas se encarga de proporcionar una base sólida sobre la que
construir los diferentes proyectos, automatizando la creación de proyectos siguiendo 
las mejores prácticas y proporcionando una base sólida sobre la que construir. Estos
componentes y plantillas se construyen de forma modular y se empaquetan dentro de
JFrog Artifactory \cite{Artifactory}, un repositorio de artefactos que se encarga de almacenar y distribuir
los diferentes paquetes de código. Por último, la documentación del proyecto se encarga de 
proporcionar una guía clara y detallada sobre el uso de las herramientas y componentes, así 
como de los procesos y metodologías empleadas en el desarrollo del proyecto. Esta documentación
está desarrollada utilizando Astro \cite{Astro} y se almacena en el repositorio donde se encuentran los
componentes y plantillas. \medskip


\subsection{Infraestructura y herramientas}
\subsubsection{Descripción general de la infraestructura}
La infraestructura y las herramientas son la base sobre la que se construirán
los diferentes proyectos de aprendizaje automático. Se encargan de proporcionar
un entorno de desarrollo e investigación eficiente, que permita a los miembros
del equipo centrarse en el desarrollo de modelos sin tener que preocuparse por
la configuración. Concretamente, se han desplegado dos plataformas 
que vienen a cubrir varias de las necesidades fundamentales de los proyectos
como son la gestión y exploración de datasets, la monitorización de experimentos o
el almacenamiento de modelos de inteligencia artificial. Además, se ha añadido un sistema de autenticación 
para garantizar la seguridad y privacidad de los datos.\medskip

Esta infraestructura se ha desplegado en un servidor interno de la empresa
utilizando contenedores de Docker. La elección de esta tecnología se debe a
que permite la creación de entornos aislados y portables, lo que facilita el
despliegue de las aplicaciones. Se ha utilizado Docker Compose para
sincronizar el despliegue de los diferentes servicios, lo que permite
realizar despliegues automatizados mediante las acciones de GitLab CI. La
figura \ref{fig:internal-server} muestra una vista general de la infraestructura
desplegada en el servidor interno de la empresa, donde se pueden observar
como se integran los diferentes servicios y sus respectivas conexiones. Además,
se puede observar que todos los servicios están interconectados mediante un
proxy inverso mediante Nginx, que se encarga de redirigir las peticiones en 
función de la URL. Esto permite que todos los servicios sean
accesibles desde el exterior a través de un único punto de entrada y que el
sistema de autenticación sea común para todos ellos.

\begin{figure}[ht]
    \centering
    \includegraphics[width=\textwidth]{internal-server.png}
    \caption{Vista general del proyecto}\label{fig:internal-server}
\end{figure}

\subsubsection{Plataformas integradas}
Previo a la elección de las plataformas integradas, se realizó una evaluación
a nivel de equipo para determinar las necesidades que se debían cubrir. Se
identificaron las siguientes funcionalidades fundamentales:

\begin{itemize}
    \item F1: Versionado y almacenamiento de dataset.
    \item F2: Monitorización de experimentos.
    \item F3: Almacenamiento de modelos de inteligencia artificial.
    \item F4: Integración de métricas en datasets y visualización de resultados.
    \item F5: Exploración de datasets.
\end{itemize}

Una vez identificadas las necesidades, se consensuó un criterio de selección
para las plataformas integradas. Este criterio es un criterio de mínimos, es
decir, se seleccionarán las plataformas que cumplan con el criterio mínimo
establecido y que, además, ofrezcan funcionalidades adicionales que puedan
ser de utilidad para el equipo. El criterio de selección se basa en los
siguientes aspectos: 

\begin{itemize}
    \item \textbf{C1 (Facilidad de uso):} Se valora muy positivamente la facilidad de uso de las
    plataformas, ya que se considera que no todo el equipo no tiene experiencia
    previa en el uso de estas herramientas.
    \item \textbf{C2 (Integración con otras herramientas):} Es fundamental que las
    plataformas integradas sean compatibles con las librerías y herramientas
    que se utilizan generalmente en proyectos (TensorFlow, PyTorch, etc.).
    \item \textbf{C3 (Poca dependencia sobre la infraestructura):} Medimos la dependencia
    sobre una plataforma como el numero de acciones que se deben realizar para
    migrar un proyecto vanilla, es decir, un proyecto que no ha sido
    desarrollado con la plataforma en mente. Y penalizando aquellas practicas
    que sean propias de la plataforma y que no sean comunes en la industria.
\end{itemize}

Con estos criterios en mente, se tuvieron en cuenta las siguientes
plataformas a la hora de realizar la evaluación: MLflow, ClearML, Kedro, ZenML, Data Version Controller (DVC),
Rath, Apache Superset. Cada una de estas plataformas tiene diferentes enfoques y 
funcionalidades, pero todas ellas cubren una o varias de las necesidades fundamentales
identificadas, por lo que se consideraron candidatas para su integración en la
infraestructura. La infraestructura final se compondrá de una o varias de estas
plataformas en función de los criterios previamente establecidos. A continuación, se 
muestra un análisis detallado de las plataformas evaluadas y las funcionalidades que ofrecen.

\begin{itemize}
    \item \textbf{MLflow:} MLflow es una plataforma MLOPs de código abierto para la gestión del ciclo de vida de
    los modelos. Ofrece una interfaz de usuario para el seguimiento de experimentos, la gestión de modelos 
    y la implementación de modelos en diferentes entornos. MLflow es compatible con la mayor parte de librerías de aprendizaje 
    automático, como TensorFlow o PyTorch. Uno de los puntos fuertes de MLflow es su gran comunidad, ya que es
    una de las plataformas más utilizadas en la industria. Sin embargo, no ofrece funcionalidades relacionadas con
    la gestión, evaluación o versionado de datasets ni con la exploración de los mismos. La dependencia sobre la
    plataforma varía en función de la tarea que se quiera realizar, pero en general, es una plataforma que no
    ata al usuario a su ella. La documentación se queda corta en cuestión de claridad y ejemplos, lo que puede
    dificultar la adopción de la plataforma por parte de los miembros del equipo.
    \item \textbf{ClearML:} ClearML al igual que MLflow, es una plataforma MLOps de código abierto que ofrece
    las mismas funcionalidades que MLflow en relación a la gestión a la gestión de experimentos, pero con la
    diferencias, que este si cuenta con funcionalidades relacionadas con la gestión, evaluación o versionado de
    datasets. ClearML también es compatible con la mayor parte de librerías populares aunque no tantas ni tan
    variadas como MLflow, pero ofrece una API que permite de la integración de estas de forma manual. La
    dependencia sobre la plataforma es mínima, ya que con pocos cambios se pueden lanzar experimentos sobre
    un código base. La documentación es clara y ofrece ejemplos en formato tanto de texto como de video, con
    proyectos sencillos y claros que permiten entender rápidamente el funcionamiento de la plataforma. El 
    principal punto débil de ClearML es que no cuenta con una gran comunidad, lo que puede dificultar a
    la hora de encontrar soluciones a ciertas problemáticas. Otro de sus puntos débiles es que por defecto
    no cuenta con un sistema de autenticación robusto, lo que te obliga a implementar uno por tu cuenta.
    \item \textbf{Data Version Controller (DVC):} DVC y DVC Studio son dos herramientas de código abierto que
    están diseñadas para manejar grandes volúmenes de datos, modelos y experimentos. DVC
    se centra en el versionado y almacenamiento de datasets, mientras que DVC Studio se centra en la
    monitorización de experimentos, visualización de resultados y almacenamiento de modelos. Además, 
    DVC Live proporciona integraciones con un número considerable de librerías de aprendizaje automático.
    La documentación está bien estructurada aunque no es tan clara como la de ClearML, pero cuenta con una
    comunidad bastante activa. En cuanto a los aspectos negativos de DVC, la principal desventaja es que
    la curva de aprendizaje es bastante pronunciada, lo que dificulta su adopción. Otro punto en contra
    es la dependencia gigantesca que tienen los proyectos que usan DVC, ya que se necesita de muchos archivos
    de configuración, integraciones manuales dentro del código y un dominio completo de los comandos de la
    herramienta para poder trabajar con ella.
    \item \textbf{Kedro:} TODO:
    \item \textbf{ZenML:} TODO:
    \item \textbf{Rath:} TODO:
    \item \textbf{Apache Superset:} TODO:
    \item \textbf{Grafana:} TODO:
\end{itemize}

Para finalizar con la evaluación, se ha realizado una tabla comparativa que muestra
las funcionalidades y criterios que cumple cada una de ellas de forma resumida.
\begin{table}[ht]
    \centering 
    \begin{tabular}{lccccccccc}  
        
        \toprule
        \multirow{2}{*}{\parbox[c]{.2\linewidth}{\centering Tecnología}} & 
        \multicolumn{5}{c}{\textbf{Funcionalidades}} && 
        \multicolumn{3}{c}{\textbf{Criterios}} \\ 
        
        \cmidrule{2-6} \cmidrule{8-10}
        & {\centering F1} & {F2} & {F3}& {F4} & {F5} && {C1} & {C2} & {C3}\\
        
        \midrule
        MLflow           & --     & \check & \check & --     & --     && \check & \check & \check \\
        ClearML          & \check & \check & \check & \check & --     && \check & \check & \check \\
        DVC              & \check & \check & \check & \check & --     && --     & \check & --     \\ 
        Kedro            & --     & --     & --     & --     & --     && --     & --     & --     \\  
        ZenML            & --     & --     & --     & --     & --     && --     & --     & --     \\ 
        Rath             & --     & --     & --     & --     & \check && \check & --     & \check \\ 
        Apache Superset  & \check & --     & --     & \check & --     && --     & --     & \check \\ 
        Grafana          & \check & --     & --     & \check & --     && --     & --     & \check \\ 
        \bottomrule
        
    \end{tabular}
    \caption{Tabla comparativa de las plataformas evaluadas}
    \label{tab:comparative-table} 
\end{table}


\subsubsection{Herramientas de desarrollo}
\subsubsection{Seguridad y priviacidad}
\subsubsection{Despligue Automatizado}
\subsection{Catalogo de componentes}
El catalogo de componentes y plantillas es una herramienta de gestión
de conocimiento que permite a los integrantes de un equipo compartir,
reutilizar y colaborar en la creación de estándares para una mayor
eficiencia en el desarrollo de modelos de IA. Los componentes son
elementos que representan pequeñas funcionalidades dentro de un
proyecto que pueden ser reutilizados de una forma sencilla. Las plantillas 
por otro lado son estructuras más complejas que agrupan varios componentes, 
configuraciones y reglas de negocio para crear un modelo dentro de una 
temática específica por cada una de ellas. A todos estos elementos los hemos
denominado bajo el nombre de STAC (Simple Tecnalia AI Components).\medskip

\subsubsection{Adaptación del diseño atómico}
La metodología de diseño atómico es una técnica que se basa en la
creación de componentes, que se pueden reutilizar en diferentes partes
de un proyecto. En el desarrollo de modelos de IA se puede adaptar
esta técnica para crear componentes que representen funcionalidades
específicas, como el preprocesamiento de datos, la selección de
características, la evaluación de modelos, entre otros.\medskip

Como hemos mencionado anteriormente, los componentes se dividen
a su vez en multitud de categorías (atómicos, moleculares, organismos, plantillas, etc.)
esta división tiene sentido dentro de su concepción original pero
en el caso de los modelos de IA, la división de los componentes se puede hacer
de una forma más sencilla, ya que en el caso de la investigación en IA, una
abstracción más sencilla puede ser más útil para los investigadores. Es por
ello que se propone una división de los componentes en tres categorías:

\begin{itemize}
    \item \textbf{Componentes atómicos:} son los componentes más sencillos
    que representan una funcionalidad específica.
    \item \textbf{Componentes compuestos:} estos componentes agrupan
    varios componentes atómicos para realizar una funcionalidad más
    compleja.
    \item \textbf{Plantillas:} estructuras de proyectos completan que buscan
    solucionar un problema específico utilizando una técnica concreta. Están
    formadas a su vez por componentes compuestos y atómicos. Además, las
    plantillas también agrupan configuraciones, reglas de negocio y
    diversas integraciones con otros sistemas.
\end{itemize}

Todos los componentes y plantillas se almacenan en un repositorio
compartido, donde los integrantes del equipo pueden colaborar en la
creación de nuevos componentes y plantillas, así como en la mejora de
los existentes.

\subsubsection{Estructura del sistema de componentes}
La arquitectura que se ha decidido implementar para el sistema de componentes
es conocida como monorepo multi-paquete. Un monorepo es una práctica de 
desarrollo de software donde todos los proyectos relacionados 
se almacenan en un único repositorio de código fuente. Esto significa que en 
lugar de tener múltiples repositorios para cada uno de los componentes, todo se 
mantienen en un único lugar. Por otro lado, que sea multi-paquete significa que
que los diferentes elementos del monorepo se organizan mediante paquetes de
software, lo que facilita su distribución de forma independiente. Este tipo de 
arquitectura cuenta con varias ventajas, entre las que se encuentran:

\begin{itemize}
    \item \textbf{Facilidad de gestión:} Tener todo en un solo lugar simplifica la gestión 
    del código, las dependencias y las versiones. No es necesario alternar entre
    diferentes repositorios para hacer cambios o resolver problemas.
    \item \textbf{Consistencia:} Todos los proyectos dentro del monorepo pueden seguir 
    las mismas convenciones, estructura de carpetas, y configuraciones, 
    lo que garantiza una mayor consistencia en el código. Esto permite que el código
    sea más fácil de mantener a largo plazo.
\end{itemize}

Uno de los principales problemas que puede surgir al utilizar un monorepo es la
complejidad que puede acarrear la organización de las diferentes carpetas, ya
que al tener todo en un solo lugar, la cantidad de archivos puede llegar
a ser muy grande y contar con un nivel de anidamiento muy profundo. Para evitar este
problema, se ha decidido tomar una estructura de carpetas basada en 
\textit{Screaming architecture}, una arquitectura de software que busca anteponer la
lógica de negocio sobre las partes técnicas del sistema. En este caso, nuestra lógica de
negocio se relaciona en torno a el problema que se busca resolver, es decir los diferentes 
conjuntos de problemas de las series temporales.\medskip

La figura \ref{fig:screaming-arch} muestra un ejemplo simplificado de la estructura
de carpetas del proyecto. Por cada uno de los diferentes problemas, se ha creado una 
carpeta principal, que a su vez contiene diferentes subcarpetas que agrupan los 
diferentes componentes dentro de cada una de las temáticas (procesamiento, modelo, métricas). 
En caso de que se necesite añadir un nuevo componente, simplemente se crea una nueva
carpeta dentro de la temática correspondiente. Por ejemplo, tomando la temática preprocesamiento de datos
dentro de clasificación, se buscaría la carpeta \textit{Data-Processing} dentro
\textit{TimeSeries-Classification} y en caso de no existir, se crearía una nueva carpeta, y dentro de ella 
se añadiría el nuevo componente. \medskip

Lo que se consigue con este enfoque es que cada uno de los diferentes problemas cuente con una 
estructura similar pero que a su vez tengan la posibilidad de incluir temáticas propia. En la
figura \ref{fig:screaming-arch} se puede ver como cada uno de los problemas cuenta con componentes
de procesamiento de datos, pero que a su vez, hay algunos de ellos que cuentan con componentes
específicos de modelos o métricas. Esto se puede deber a que cada uno de los problemas tiene
una necesidades específicas o que no se han encontrado componentes que se puedan reutilizar
en este momento.

\begin{figure}[ht]
    \dirtree{%
        .1 /components.
        .2 /TimeSeries-Forecasting.
        .3 /Data-Ingestion.
        .4 ingestion-component-1.
        .4 ingestion-component-2.
        .3 /Data-Processing.
        .4 processing-component-1.
        .3 /Models.
        .4 model-component-1.
        .2 /TimeSeries-Classification.
        .3 /Data-Processing.
        .4 processing-component-2.
        .3 /Metrics.
        .4 metric-component-1.
        .2 /TimeSeries-AnomalyDetection.
        .3 /Data-Processing.
        .4 processing-component-3.
    }
    \caption{Ejemplo de estructura de carpetas basada en Screaming Architecture.}
    \label{fig:screaming-arch}
\end{figure}

\subsubsection{Empaquetado de componentes}
La idea de los componentes es que sean fácilmente integrables en cualquier
proyecto de una forma sencilla. Para ello, se ha decidido empaquetar cada
componente de forma independiente y distribuirlos dentro de un repositorio
privado, de forma que se puedan instalar utilizando pip o cualquier otro 
gestor de paquetes como poetry.\medskip

Dentro de nuestro caso de uso, queremos tener la posibilidad de instalar
solo aquellos componentes que necesitemos en cada momento, sin tener que
descargarnos todo el contenido del repositorio. Esto es especialmente 
importante ya que las dependencias de las librerías de IA pueden llegar a
ser muy pesadas. A su vez, también se quiere que todos nuestros componentes
hereden del mismo nombre de paquete, para que se puedan utilizar de una forma
transparente a la hora de importarlos en un proyecto. Es decir, si tenemos por
ejemplo un componentes de ingesta de datos llamado \textit{get\_data} y otro de 
visualización que imprime una gráfica llamado \textit{show\_graph}, la forma de 
importarlo en un proyecto sería la siguiente:

\begin{verbatim}
    $ pip install stac-show-graph stac-get-data

    >> from stac.visualization.show_graph import show_graph
    >> from stac.data_ingestion.get_data import get_data
    >> data = get_data()
    >> show_graph(data)
\end{verbatim}

Para conseguir este efecto es necesario compendrer como funciona el empaquetado
de librerías en Python. En Python, un paquete es una carpeta que contiene por lo menos
un archivo \textit{\_\_init\_\_.py} y un \textit{pyproject.toml} o \textit{setup.py} . 
La forma en la que python se importan los paquetes es a través de un sistema de rutas, 
donde se toma la ruta hacia el paquete desde el directorio raíz del proyecto. En la
figura \ref{fig:min-package} se puede ver como es el contenido de cada componente.

\begin{figure}[ht]
    \dirtree{%
        .1 /component-name.
        .2 pyproject.toml.
        .2 /stac/component-category.
        .3 \_\_init\_\_.py.
    }
    \caption{Estructura mínima de un componente empaquetado.}
    \label{fig:min-package}
\end{figure}

Esta estructura se puede repetir para cada uno de los componentes que se quieran
añadir al repositorio. No obstante, y aprovechando que cada componente está separada
en un proyecto independiente, se pueden añadir más utilidades como tests, documentación
o ejemplos de uso. Como estamos utilizando poetry para gestionar las dependencias, y
las configuraciones globales del proyecto, también se van a añadir sus archivos correspondientes. 

\begin{figure}[ht]
    \dirtree{%
        .1 /component-name.
        .2 README.md.
        .2 pyproject.toml.
        .2 poetry.lock.
        .2 Makefile.
        .2 /tests.
        .2 /stac/component-category.
        .3 \_\_init\_\_.py.
    }
    \caption{Estructura final de un componente empaquetado.}
    \label{fig:real-package}
\end{figure}

Este empaquetado puede parecer tedioso si se tiene que hacer manualmente, pero en
futuras secciones veremos como se puede automatizar este proceso para que sea lo más
sencillo posible. Es muy importante que la estructura sea la descrita para optimizar
la experiencia de desarrollo de los integrantes del equipo.

\subsubsection{Sistema de plantillas}
La herramienta que hemos utilizado para la creación de plantillas es Cookiecutter.
Cookiecutter es una utilidad para la generación de proyectos que te permite inicializar
un proyecto a partir de plantillas predefinidas. Funciona siguiendo un principio 
básico: en lugar de empezar un nuevo proyecto desde cero y tener que configurar 
todo manualmente, puedes usar una plantilla predefinida que ya incluya la estructura 
de directorios, archivos básicos, configuraciones iniciales, y cualquier otro 
componente necesario para tu proyecto.\medskip

Esto es especialmente útil en entornos donde necesitas crear múltiples proyectos 
con una estructura similar donde puede haber proyectos con requisitos comunes, 
como la configuración de un marco de trabajo específico, estructura de directorios 
estándar, o incluso configuraciones de pruebas y documentación.\medskip

\subsubsection{Integración continua y despliegue continuo}

\subsection{Documentación del proyecto}
La documentación juega un papel crucial en este proyecto, ya 
que proporciona una guía detallada sobre cómo utilizar los distintos 
componentes y plantillas disponibles. En este contexto, la adopción 
de Astro como herramienta para la generación de documentación
automática se debe a su facilidad de uso, ya que permite crear
paginas de forma sencilla utilizando Markdown, lo que facilita la
creación de documentos por los distintos integrantes del equipo.
Ya existen proyectos que han utilizado Astro para la generación
de documentación, como Microsoft con su proyecto Fluent UI que
utiliza Astro.

Concretamente, el proyecto usa Starlight, una plantilla de Astro que
proporciona una estructura de documentación predefinida, lo que
facilita la creación de documentación de forma rápida y sencilla.
Además, Starlight incluye un sistema de búsqueda y organización
de documentación, lo que facilita la navegación y búsqueda de
información en la documentación. Otra de sus ventajas es la
facilidad de customización, ya que permite modificar la apariencia
utilizando CSS o Tailwind e incluso incorporar funcionalidades
propias mediante JavaScript.

\subsubsection{Guías y manuales}
Para facilitar la integración de sistemas complejos como el aquí
propuesto, es necesario proporcionar guías que expliquen
de forma detallada cómo se debe interactuar con los diferentes
procesos. En este sentido, la documentación del proyecto incluye
un apartado de guías y manuales que proporciona esa información
de forma detallada. Además, se incluyen ejemplos de uso y casos
prácticos que ayudan a entender cómo se deben utilizar los distintos
elementos de forma interactiva y dinámica.

En este momento, la documentación incluye las siguientes guías:
\begin{itemize}
    \item Guía de introducción: proporciona una visión general del
    proyecto y que partes lo componen.
    \item Guía de instalación: explica cómo instalar las diferentes 
    herramientas en local y conseguir acceso a los servicios. 
    \item Guía de plantillas y componentes: proporciona información detallada sobre cómo
    utilizar los distintos componentes y plantillas disponibles.
    \item Guía de ClearMl: introducción a la plataforma MLOps ClearMl y cómo
    utilizarla en el proyecto.
    \item Guía de contribución: proporciona información sobre cómo
    contribuir al proyecto y cómo colaborar con otros miembros del
    equipo.
\end{itemize}

En el futuro, se espera ampliar la documentación con nuevas guías
que no solo expliquen cómo utilizar los distintos elementos sino que
también se centren en aspectos enfocados a la IA y el aprendizaje
automático, como por ejemplo guías sobre cómo solucionar problemas
relacionados con el día a día de la empresa.

\subsubsection{Construcción automática de documentación}


\subsubsection{Documentación de componentes y plantillas}

\subsubsection{Sistema de búsqueda y organización de documentación}



\pagebreak