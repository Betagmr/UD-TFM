\subsection{Sistema de componentes}
\subsubsection{Diseño Atómico}
El diseño atómico es una metodología de diseño que se centra en la creación
de sistemas de diseño modulares y reutilizables. La idea principal es dividir
las diferentes funcionalidades de un sistemas en sus partes más fundamentales,
de manera que cada una de estas partes pueda ser reutilizada en diferentes
contextos. Este enfoque permite tener un mayor control sobre cada una de las
partes del sistema, facilitando su mantenimiento, documentación y reutilización.
Originalmente, el diseño atómico ha sido aplicado en el diseño de interfaces
de usuario, pero su filosofía puede ser aplicada a cualquier sistema de diseño
modular. En el contexto de este proyecto, el diseño atómico se aplicará al
diseño de un sistema de componentes para el desarrollo de modelos de aprendizaje
automático.