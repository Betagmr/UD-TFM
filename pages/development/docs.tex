\subsection{Documentación del proyecto}
La documentación juega un papel crucial en este proyecto, ya 
que proporciona una guía detallada sobre cómo utilizar los distintos 
componentes y plantillas disponibles. En este contexto, la adopción 
de Astro como herramienta para la generación de documentación
automática se debe a su facilidad de uso, ya que permite crear
paginas de forma sencilla utilizando Markdown, lo que facilita la
creación de documentos por los distintos integrantes del equipo.
Ya existen proyectos que han utilizado Astro para la generación
de documentación, como Microsoft con su proyecto Fluent UI que
utiliza Astro.

Concretamente, el proyecto usa Starlight, una plantilla de Astro que
proporciona una estructura de documentación predefinida, lo que
facilita la creación de documentación de forma rápida y sencilla.
Además, Starlight incluye un sistema de búsqueda y organización
de documentación, lo que facilita la navegación y búsqueda de
información en la documentación. Otra de sus ventajas es la
facilidad de customización, ya que permite modificar la apariencia
utilizando CSS o Tailwind e incluso incorporar funcionalidades
propias mediante JavaScript.

\subsubsection{Guías y manuales}
Para facilitar la integración de sistemas complejos como el aquí
propuesto, es necesario proporcionar guías que expliquen
de forma detallada cómo se debe interactuar con los diferentes
procesos. En este sentido, la documentación del proyecto incluye
un apartado de guías y manuales que proporciona esa información
de forma detallada. Además, se incluyen ejemplos de uso y casos
prácticos que ayudan a entender cómo se deben utilizar los distintos
elementos de forma interactiva y dinámica.

En este momento, la documentación incluye las siguientes guías:
\begin{itemize}
    \item Guía de introducción: proporciona una visión general del
    proyecto y que partes lo componen.
    \item Guía de instalación: explica cómo instalar las diferentes 
    herramientas en local y conseguir acceso a los servicios. 
    \item Guía de plantillas y componentes: proporciona información detallada sobre cómo
    utilizar los distintos componentes y plantillas disponibles.
    \item Guía de ClearMl: introducción a la plataforma MLOps ClearMl y cómo
    utilizarla en el proyecto.
    \item Guía de contribución: proporciona información sobre cómo
    contribuir al proyecto y cómo colaborar con otros miembros del
    equipo.
\end{itemize}

En el futuro, se espera ampliar la documentación con nuevas guías
que no solo expliquen cómo utilizar los distintos elementos sino que
también se centren en aspectos enfocados a la IA y el aprendizaje
automático, como por ejemplo guías sobre cómo solucionar problemas
relacionados con el día a día de la empresa.

\subsubsection{Construcción automática de documentación}


\subsubsection{Documentación de componentes y plantillas}

\subsubsection{Sistema de búsqueda y organización de documentación}

