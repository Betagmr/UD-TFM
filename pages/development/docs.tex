\subsection{Documentación del proyecto}
La documentación juega un papel crucial en este proyecto, ya 
que proporciona una guía detallada sobre cómo utilizar los distintos 
componentes y plantillas disponibles. En este contexto, la adopción 
de Astro como herramienta para la generación de documentación
automática se debe a su facilidad de uso, ya que permite crear
paginas de forma sencilla utilizando Markdown, lo que facilita la
creación de documentos por los distintos integrantes del equipo.
Ya existen proyectos que han utilizado Astro para la generación
de documentación, como Microsoft con su proyecto Fluent UI que
utiliza Astro.\medskip

Concretamente, el proyecto usa Starlight, una plantilla de Astro que
proporciona una estructura de documentación predefinida, lo que
facilita la creación de documentación de forma rápida y sencilla.
Además, Starlight incluye un sistema de búsqueda y organización
de documentación, lo que facilita la navegación y búsqueda de
información en la documentación. Otra de sus ventajas es la
facilidad de customización, ya que permite modificar la apariencia
utilizando CSS o Tailwind e incluso incorporar funcionalidades
propias mediante JavaScript.

\subsubsection{Guías y manuales}
Para facilitar la integración de sistemas complejos como el aquí
propuesto, es necesario proporcionar guías que expliquen
de forma detallada cómo se debe interactuar con los diferentes
procesos. En este sentido, la documentación del proyecto incluye
un apartado de guías y manuales que proporciona esa información
de forma detallada. Además, se incluyen ejemplos de uso y casos
prácticos que ayudan a entender cómo se deben utilizar los distintos
elementos de forma interactiva y dinámica.

En este momento, la documentación incluye las siguientes guías:
\begin{itemize}
    \item \textbf{Guía de introducción:} proporciona una visión general del
    proyecto y que partes lo componen.
    \item \textbf{Guía de instalación:} explica cómo instalar las diferentes 
    herramientas en local y conseguir acceso a los servicios. 
    \item \textbf{Guía de plantillas y componentes:} proporciona información detallada sobre cómo
    utilizar los distintos componentes y plantillas disponibles.
    \item \textbf{Guía de ClearMl:} introducción a la plataforma MLOps ClearMl y cómo
    utilizarla en el proyecto.
    \item \textbf{Guía de contribución:} proporciona información sobre cómo
    contribuir al proyecto y cómo colaborar con otros miembros del
    equipo.
\end{itemize}

En el futuro, se espera ampliar la documentación con nuevas guías
que no solo expliquen cómo utilizar los distintos elementos sino que
también se centren en aspectos enfocados a la IA y el aprendizaje
automático, como por ejemplo guías sobre cómo solucionar problemas
relacionados con el día a día de la empresa.

\subsubsection{Sistema de búsqueda y organización de documentación}
El poder encontrar información de forma rápida y sencilla es crucial
en un proyecto de estas características, ya que la documentación
puede llegar a crecer significativamente y ser difícil encontrar
la información necesaria. Una de las ventajas de Astro con Starlight
es que proporciona de forma automatizada un sistema de búsqueda 
que permite realizar búsquedas en base a palabras clave.\medskip

El sistema que utiliza por detrás se llama Pagefind, el cual es gratuito 
y de código abierto. Pagefind es un motor de búsqueda de documentos
que entre sus características incluye no solo la búsqueda por palabras
clave sino también el filtrado de resultados mediante etiquetas. Esta 
funcionalidad es clave y permiten encontrar la información de forma más precisa. 
Para poder indexar los documentos, Pagefind necesita que estos cuenten con
unas etiquetas especiales que le permitan identificar que cierto documento
pertenece a una categoría en concreto. En este sentido, ciertas páginas
de la documentación cuenta con una sección llamada \textit{Tags} en la que
se incluyen las etiquetas correspondientes.

% TODO: Añadir imagen del sistema de búsqueda
% TODO: Añadir imagen de los tags

\subsubsection{Construcción automática de documentación}
Como se ha mencionado anteriormente, cada componente y plantilla 
viene con su propia documentación, la cual se genera dentro de su
respectiva carpeta mediante un archivo Readme.md. La decision de 
que cada componente tenga su documentación en su propia carpeta
es para facilitar la experiencia de desarrollo ya que todos los
elementos de cada componente se encuentran en un mismo lugar y no
es necesario buscar en diferentes carpetas para encontrar la
información necesaria.\medskip

Esto pese a ser una ventaja, también genera un problema a la hora de
generar la documentación, ya que es necesario un sistema que sea
capaz de recorrer todas las carpetas, extraer los archivos Readme.md
y Readme.mdx de cada una de ellas y generar la respectiva página
de documentación. Para solucionar este problema, se ha creado un
script de bash que recorre las carpetas que contienen componentes o
plantillas y copia estos archivos en una carpeta con el nombre de
su carpeta original con el fin de evitar problemas de colisiones
entre los nombres de los diferentes archivos. Una vez hecho esto,
se ejecuta el comando astro build y se sube la documentación a la
plataforma de GitLab Pages.

\subsubsection{Documentación de componentes y plantillas}
