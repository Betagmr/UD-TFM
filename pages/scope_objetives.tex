\section{Objetivos y alcance}
En esta sección se introducen los objetivos del 
proyecto, habiéndose realizado una division entre el principal y los
secundarios. Ademas de ello, se presentan los elementos que forman el alcance
y los que quedan fuera del mismo.

\subsection{Objetivos generales}
El objetivo principal del proyecto es diseñar e implementar un estándar tecnológico y 
operacional que cubra las necesidades más comunes dentro de un equipo de \textit{data science}. Se 
busca agilizar los tiempos de desarrollo y estandarizar los procesos, con el fin de
facilitar la colaboración entre investigadores y la reutilización del conocimiento. A
continuación, se detallan los objetivos específicos que guiarán el desarrollo:

\begin{itemize}
    \item \textbf{Agilizar el proceso inicial de proyectos:} Optimizar las primeras etapas 
    de los proyectos, identificando y eliminando aquellos procesos repetitivos que no aportan
    valor y que puedan retrasar su puesta en marcha. 
    \item \textbf{Facilitar la colaboración entre investigadores:} Implementar herramientas y 
    métodos que fomenten una cooperación fluida y efectiva entre los miembros del equipo de 
    investigación, con el fin de potenciar la sinergia y aprovechar al máximo el conocimiento 
    colectivo.
    \item \textbf{Definir procesos mediante buenas prácticas:} Establecer un marco de trabajo 
    basado en buenas prácticas de gestión de proyectos, con el objetivo de estandarizar los 
    procesos y garantizar su eficiencia y calidad.
    \item \textbf{Automatizar el desarrollo de modelos robustos:} Investigar y aplicar 
    técnicas que contribuyan al desarrollo automático de modelos de aprendizaje automático,
    con el fin de reducir el tiempo y el esfuerzo necesarios para obtener resultados de calidad.
    \item \textbf{Promover la reutilización del conocimiento:} Desarrollar mecanismos y herramientas 
    que faciliten la captura, organización y difusión del conocimiento generado durante el desarrollo 
    de los proyectos, con el propósito de fomentar su reutilización en futuras investigaciones y 
    actividades relacionadas.
\end{itemize}

El cumplimiento de estos objetivos se espera que no solo mejore la eficiencia y la calidad de los 
proyectos, sino que también contribuya a la creación de un entorno de trabajo más colaborativo y
enriquecedor para los miembros del equipo de investigación.

\subsection{Alcance}
En esta sección se definen los límites del proyecto, estableciendo lo que está
incluido y excluido dentro mismo. Se describirá de manera detallada las
actividades que forman parte del desarrollo final, así como aquellos elementos
que no están incluidos en el alcance del proyecto. Aunque el enfoque de este 
proyecto podría aplicarse a una amplia variedad de problemas en el ámbito del 
aprendizaje automático, en el contexto de este TFM nos centraremos en tres de 
los casos más comunes dentro del marco de las series temporales: forecasting, 
clasificación y detección de anomalías. A continuación, se detallan las 
actividades que forman parte del alcance del proyecto.

\subsubsection{Dentro del alcance}
\begin{itemize}
    \item \textbf{Integración de sistemas externos:} Se incluirá la configuración de sistemas 
    externos, como plataformas MLOPs o herramientas de visualización, con las plantillas 
    de proyectos base. Esto permitirá una integración más fluida y rápida de estos sistemas con 
    los proyectos, facilitando el flujo de datos y la visualización de resultados.
    \item \textbf{Plantillas de proyectos base:} Desarrollar plantillas para 
    los tres problemas de series temporales comentados anteriormente. Estas plantillas 
    servirán como punto de partida para proyectos específicos dentro de cada uno de estos dominios
    y definirán desde el principio una estructura y un conjunto de herramientas comunes. Además,
    se incluirán ejemplos de código y documentación que faciliten su uso y comprensión.
    \item \textbf{Componentes esenciales:} Identificar y almacenar los componentes 
    esenciales de cada proyecto, incluyendo modelos, algoritmos, métricas de evaluación y
    preprocesamiento de datos. Estos componentes se almacenarán y documentarán de forma
    que puedan ser reutilizados en futuros proyectos, facilitando la transferencia de conocimiento.
    \item \textbf{Proceso de AutoML:} Diseñará y ejecutar procesos de AutoML 
    (Machine Learning Automatizado) que demostrará cómo se pueden combinar los conocimientos 
    adquiridos de todos los proyectos para desarrollar un sistema de aprendizaje automático 
    automatizado. Este proceso utilizará las plantillas y componentes esenciales almacenados 
    para generar modelos de forma automática.
\end{itemize}

\subsubsection{Fuera del alcance}
\begin{itemize}
    \item \textbf{Desarrollo de modelos específicos:} Aunque se incluirán ejemplos de
    modelos y algoritmos, el desarrollo de modelos específicos para problemas concretos
    no forma parte del alcance de este proyecto. Se espera que los modelos desarrollados
    sean generales y puedan ser adaptados a problemas específicos por los usuarios.
    \item \textbf{Despliegue de modelos:} El despliegue de modelos en producción no forma
    parte del alcance de este proyecto. Se espera que los modelos desarrollados puedan ser
    desplegados en sistemas de producción, pero no se incluirá en este proyecto. El enfoque
    se centrará exclusivamente en el desarrollo de los mismos.
\end{itemize}

\pagebreak