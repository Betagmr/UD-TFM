\section{Objetivos y alcance}
En esta sección se introducen los objetivos en los que va consistir el
proyecto, habiéndose realizado una division entre el principal y los
secundarios. Ademas de ello, se presentan los elementos que forman el alcance,
asi como se comentaran otros que no forman parte de este.

\subsection{Objetivos generales}
El objetivo principal del proyecto es diseñar e implementar un estándar tecnológico y 
operacional que cubra todas las necesidades de un equipo de data science. Este estándar
deberá permitir la implementación de prácticas modernas de MLOps, CI/CD y metodologías
ágiles, facilitando la transición y reduciendo la complejidad asociada. A continuación,
se detallan los objetivos específicos que guiarán el desarrollo:

\begin{itemize}
    \item \textbf{Identificación de necesidades:} Antes de comenzar con el proceso de
    diseño, se debe realizar un análisis de las necesidades del equipo para poder
    identificar que requisitos debe cumplir el stack tecnológico.
    \item \textbf{Investigación del estado del arte:} Una vez identificadas las necesidades
    del equipo, se debe realizar una investigación exhaustiva de las tecnologías, herramientas
    y plataformas que se encuentran disponibles en el mercado.
    \item \textbf{Diseño del estándar tecnológico:} Una vez realizada la investigación, se procederá
    a diseñar el estándar tecnológico que mejor se adapte a las necesidades del equipo. Este diseño
    busca ser simple pero efectivo, permitiendo una transición suave y una reducción de la
    complejidad asociada.
    \item \textbf{Definición de las fases de desarrollo:} Las diferentes fases que se siguen
    dentro del desarrollo del proyecto an de estar bien definidas, permitiendo una comprensión
    clara y evitando posibles malentendidos o confusiones durante su ejecución.
    \item \textbf{Despliegue de la infraestructura:} Acorde al diseño realizado, se procederá
    a desplegar la infraestructura necesaria para el funcionamiento correcto de los diferentes
    flujos de trabajo.
    \item \textbf{Creación de documentación:} El proyecto debe ir acompañado de una documentación
    clara que permita a cualquier miembro del equipo transicionar de manera efectiva.
\end{itemize}

\subsection{Alcance}
En esta sección se definen los límites del proyecto, estableciendo lo que está
incluido y excluido dentro mismo. Se describirá de manera detallada las
actividades que forman parte del desarrollo final, así como aquellos elementos
que no están incluidos en el alcance del proyecto. Esta sección es fundamental
para tener una comprensión clara y evitar posibles malentendidos o confusiones
durante su ejecución.

\subsubsection{Dentro del alcance}
\subsubsection{Fuera del alcance}

\pagebreak