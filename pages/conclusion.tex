\section{Conclusiones y trabajo a futuro}
En esta sección se presentan las conclusiones obtenidas tras la finalización 
del proyecto, así como las lecciones aprendidas y los conocimientos adquiridos. 
Además, se presentan ideas o propuestas que podrían ser utilizadas o implementadas 
en el futuro para mejorar o ampliar el alcance del proyecto.

\subsection{Conclusiones}
Para poder llevar a cabo un análisis de los resultados de la aplicación
de este proyecto, se ha realizado una prueba piloto con un grupo de integrantes
de Tecnalia. Estos han estado probando la herramienta durante un periodo de tiempo
de dos semanas, y han proporcionado feedback sobre su experiencia a través de
la siguiente encuesta de satisfacción. Entre los participantes podemos encontrar
dos tipos de perfiles: investigadores e ingenieros de software. 

\subsection{Trabajo a futuro}
A continuación, se presentan las siguientes propuestas de mejora y ampliación
del proyecto que podrían ser implementadas en futuras iteraciones:

\begin{itemize}
    \item \textbf{Adaptación a visión por computador:} Actualmente, la herramienta
    está enfocada en problemas basados en series temporales, pero se podría ampliar
    su alcance para incluir problemas de visión por computador. Esto es especialmente
\end{itemize}