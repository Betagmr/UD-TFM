\section{Conclusiones y trabajo a futuro}
En esta sección se presentan las conclusiones obtenidas tras la finalización 
del proyecto, así como las lecciones aprendidas y los conocimientos adquiridos. 
Además, se presentan ideas o propuestas que podrían ser utilizadas o implementadas 
en el futuro para mejorar o ampliar el alcance del proyecto.

\subsection{Conclusiones}
Para poder llevar a cabo un análisis de los resultados de la aplicación
de este proyecto, se ha realizado una prueba piloto con un grupo de integrantes
de Tecnalia. Estos han estado probando la herramienta durante un periodo de tiempo
de dos semanas, y han proporcionado feedback sobre su experiencia a través de
la siguiente encuesta de satisfacción. Entre los participantes podemos encontrar
dos tipos de perfiles: investigadores e ingenieros de software.

\subsection{Trabajo a futuro}
A continuación, se presentan las siguientes propuestas de mejora y ampliación
del proyecto que podrían ser implementadas en futuras iteraciones:

\begin{itemize}
    \item \textbf{Adaptación del sistema a visión por computador:} Actualmente, la infraestructura 
    está enfocada en problemas basados en series temporales, pero se podría ampliar
    su alcance para incluir problemas de visión por computador. Se podría implementar
    no solo funcionalidades de preprocesamiento de imágenes, sino también incluir
    plataformas de etiquetado de imágenes populares como Cvat \cite{CVAT} o incluso conectar el
    sistema de versionado de datasets con herramientas como FiftyOne \cite{FiftyOne}.
    \item \textbf{Integración de ClearML con las gráficas de la empresa:} Para 
    optimizar el uso de recursos de cómputo, se propone conectar ClearML con las 
    gráficas (GPUs) de la empresa. Esta integración permitirá gestionar y monitorear las 
    tareas de procesamiento de manera eficiente, asegurando una distribución adecuada de las 
    cargas de trabajo. ClearML facilitará la asignación dinámica de recursos, mejorando 
    la eficiencia del procesamiento y reduciendo tiempos de espera. Maximizar la utilización de 
    los recursos de cómputo, se podría implementar un sistema de colas que divida las tareas 
    según su prioridad y requerimientos de recursos. Esto incluiría la creación de diferentes 
    colas para tareas de alta prioridad, tareas de procesamiento intensivo y tareas de 
    mantenimiento. ClearML gestionaría estas colas, asignando las tareas a las GPUs 
    disponibles y asegurando un uso equilibrado y eficiente de los recursos.
    \item \textbf{Monitoreo y análisis de rendimiento:} Implementar herramientas de monitoreo 
    que proporcionen métricas detalladas sobre el uso de los recursos, el rendimiento de las 
    tareas y la eficiencia del sistema en general. Estas herramientas permitirían identificar 
    cuellos de botella, optimizar la asignación de recursos y planificar mejoras futuras 
    basadas en datos concretos. Además, se podría implementar un sistema de alertas que
    notifique sobre posibles problemas o anomalías en el rendimiento del sistema, permitiendo
    una respuesta rápida y eficaz ante situaciones críticas.
\end{itemize}