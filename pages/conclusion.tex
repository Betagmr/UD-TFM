\section{Conclusiones y trabajo a futuro}
En esta sección se presentan las conclusiones obtenidas tras la finalización 
del proyecto, así como las lecciones aprendidas y los conocimientos adquiridos. 
Además, se presentan ideas o propuestas que podrían ser utilizadas o implementadas 
en el futuro para mejorar o ampliar el alcance del proyecto.

\subsection{Conclusiones}
Una vez finalizado el proyecto, se presentan las siguientes conclusiones en 
base a los resultados obtenidos y analizados durante el desarrollo del mismo.
Estas representan los cuatro aspectos clave del proyecto: reducción del tiempo de desarrollo, 
mejora de la productividad, reutilización del conocimiento y colaboración. Estas conclusiones 
reflejan los beneficios tangibles que el marco tecnológico propuesto aporta a los 
equipos y sus proyectos.\medskip

La implementación ha demostrado ser eficaz en la reducción significativa del tiempo 
de de\-sarro\-llo. Al automatizar tareas repetitivas y procesos manuales, se liberan 
recursos y se acelera el progreso de los proyectos. La adopción de metodologías 
ágiles y MLOps permite responder rápidamente a los cambios del entorno empresarial, 
proporcionando una ventaja competitiva crucial en cuanto a tiempo se refiere. La 
integración de plataformas como ClearML y Rath facilita la gestión y 
monitorización de experimentos, lo que reduce el tiempo dedicado a la configuración 
y gestión manual de estos procesos.\medskip

La mejora de la productividad es uno de los beneficios más destacados de la 
implementación del marco tecnológico. La automatización de tareas y la 
estandarización de procesos permiten a los equipos centrarse en actividades 
de mayor valor añadido. Además, el uso de herramientas de colaboración y la 
integración de sistemas de gestión del conocimiento aseguran que la información 
se comparta y reutilice de manera efectiva. Esto no solo incrementa la eficiencia 
operativa, sino que también mejora la calidad del trabajo, ya que los investigadores 
pueden dedicar más tiempo a la innovación y menos a tareas repetitivas.\medskip

El diseño de un sistema de componentes reutilizables y plantillas base para 
proyectos específicos facilita la reutilización del conocimiento adquirido. 
La creación de un catálogo de componentes esenciales y la documentación exhaustiva 
del proyecto aseguran que las mejores prácticas y soluciones previas se 
puedan aplicar en futuros proyectos. Este enfoque modular y estandarizado no 
solo ahorra tiempo, sino que también mejora la coherencia y calidad de los 
desarrollos, permitiendo a los equipos construir sobre trabajos anteriores 
de manera más eficiente.\medskip

La adopción de un marco común y la integración de herramientas colaborativas 
son fundamentales para mejorar la cooperación entre los miembros del equipo. 
La estandarización de procesos y la implementación de sistemas de autenticación 
y seguridad, como Keycloak, garantizan un entorno seguro y accesible para todos 
los miembros del equipo. Además, la encuesta de satisfacción realizada a los
miembros del equipo ha sido acogida de manera positiva, lo que sugiere que
existe un interés y una aceptación generalizada del nuevo marco tecnológico.\medskip

En conclusion, en base a los resultados obtenidos y analizados durante el desarrollo
del proyecto, se puede afirmar que la implementación del marco tecnológico propuesto
ha sido exitosa. Los beneficios tangibles obtenidos son muy positivos y sugieren que
el uso de este marco tecnológico puede mejorar significativamente la eficiencia y
productividad de los equipos.

\subsection{Trabajo a futuro}
A continuación, se presentan las siguientes propuestas de mejora y ampliación
del proyecto que podrían ser implementadas en futuras iteraciones:

\begin{itemize}
    \item \textbf{Adaptación del sistema a visión por computador:} Actualmente, la infraestructura 
    está enfocada en problemas basados en series temporales, pero se podría ampliar
    su alcance para incluir problemas de visión por computador. Se podría implementar
    no solo funcionalidades de preprocesamiento de imágenes, sino también incluir
    plataformas de etiquetado de imágenes populares como Cvat \cite{CVAT} o incluso conectar el
    sistema de versionado de datasets con herramientas como FiftyOne \cite{FiftyOne}.
    \item \textbf{Integración de ClearML con las tarjetas gráficas de la empresa:} Para 
    optimizar el uso de recursos de cómputo, se propone conectar ClearML con las 
    gráficas (GPUs) de la empresa. Esta integración permitirá gestionar y monitorear las 
    tareas de procesamiento de manera eficiente, asegurando una distribución adecuada de las 
    cargas de trabajo. ClearML facilitará la asignación dinámica de recursos, mejorando 
    la eficiencia del procesamiento y reduciendo tiempos de espera. Se podría implementar un 
    sistema de colas que divida las tareas según su prioridad y requerimientos de recursos. 
    Esto incluiría la creación de diferentes colas para tareas de alta prioridad, 
    tareas de procesamiento intensivo y tareas de mantenimiento. ClearML gestionaría estas colas, 
    asignando las tareas a las GPUs disponibles y asegurando un uso equilibrado y eficiente de los recursos.
    \item \textbf{Monitorización y análisis de rendimiento:} Implementar herramientas de monitorización 
    que proporcionen métricas detalladas sobre el uso de los recursos, el rendimiento de las 
    tareas y la eficiencia del sistema en general. Estas herramientas permitirían identificar 
    cuellos de botella, optimizar la asignación de recursos y planificar mejoras futuras 
    basadas en datos concretos. Además, se podría implementar un sistema de alertas que
    notifique sobre posibles problemas o anomalías en el rendimiento del sistema, permitiendo
    una respuesta rápida y eficaz ante situaciones críticas.
\end{itemize}