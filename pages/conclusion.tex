\section{Conclusiones y trabajo a futuro}
En esta sección se presentan las conclusiones obtenidas tras la finalización 
del proyecto, así como las lecciones aprendidas y los conocimientos adquiridos. 
Además, se presentan ideas o propuestas que podrían ser utilizadas o implementadas 
en el futuro para mejorar o ampliar el alcance del proyecto.

\subsection{Conclusiones}
La sección está estructurada en dos partes principales. Primero, se discuten los 
resultados basados en la experiencia directa, evaluando cómo el nuevo stack ha 
reducido los tiempos de desarrollo. Posteriormente, se presentan las opiniones 
y feedback de otros miembros del equipo que han probado y utilizado este stack, 
proporcionando una visión más amplia y diversa sobre su efectividad y beneficios.

\subsubsection{Resultados basados en los desarrollos}
Para evaluar la efectividad del stack tecnológico desarrollado, se realizaron 
varios proyectos de comparando el tiempo necesario para completarlos 
desde cero versus utilizando plantillas y componentes reutilizables. A continuación, 
se presenta una tabla con los tiempos. La tabla \ref{tabla:result-time-scores} 
muestra unicamente el tiempo de desarrollo de los proyectos, sin tener en cuenta
otros factores como el tiempo en la creación de nuevas plantillas o componentes.

\begin{table}[h!]
    \centering
    \begin{tabular}{|l|c|c|c|}
    \hline
    \textbf{Proyecto} & \textbf{Tiempo} & \textbf{Nuevos componentes} & \textbf{Utiliza plantilla} \\
    \hline
    Predicción Temperatura & 3h 30min & 5 & No \\
    \hline
    Consumo Energético & 1h 40min & 1 & Si \\
    \hline
    Clasificación Estación & 2h 00min & 2 & No \\
    \hline
    Estación SKTime & 0h 30min & 0 & Si \\
    \hline
    Anomalías Taxi & 140 & 2 & No \\
    \hline
    Temperatura AutoML & 1h 20min & 1 & Si \\
    \hline
    \end{tabular}
    \caption{Comparación de tiempos de desarrollo de proyectos}
    \label{tabla:result-time-scores}
\end{table}

La utilización de plantillas y componentes reutilizables dentro del nuevo 
marco tecnológico ha demostrado reducir significativamente los tiempos de 
desarrollo de los proyectos. Los proyectos que usaron plantillas, 
presentaron tiempos de desarrollo más bajos en comparación con aquellos que no 
las usaron. Esto sugiere que las plantillas mejora eficientemente y puede ser 
recomendado para futuros desarrollos para minimizar el tiempo invertido en proyectos.\medskip

Otro de los aspectos a destacar es que ha medida que se incorporan nuevos componentes 
al sistema de conocimiento cada vez es más complicado encontrar nuevos que aporten
valor al sistema. Este fenómeno se puede deber a que las necesidades de los proyectos
son bastante similares y los componentes que se han creado hasta ahora cubren la mayoría
de los casos de uso. Por otro lado, es posible que a medida que se vayan incorporando
nuevos miembros al sistema, estos aporten nuevas ideas que permitan la creación de nuevo
conocimiento que no se había contemplado hasta ahora.

\subsubsection{Feedback de los usuarios}
Para poder llevar a cabo un análisis de los resultados de la aplicación
de este proyecto, se ha realizado una prueba piloto con un grupo de integrantes
de Tecnalia. Estos han estado probando la herramienta durante un periodo de tiempo
de dos semanas, y han proporcionado feedback sobre su experiencia a través de
la siguiente encuesta de satisfacción. 

\subsection{Trabajo a futuro}
A continuación, se presentan las siguientes propuestas de mejora y ampliación
del proyecto que podrían ser implementadas en futuras iteraciones:

\begin{itemize}
    \item \textbf{Adaptación del sistema a visión por computador:} Actualmente, la infraestructura 
    está enfocada en problemas basados en series temporales, pero se podría ampliar
    su alcance para incluir problemas de visión por computador. Se podría implementar
    no solo funcionalidades de preprocesamiento de imágenes, sino también incluir
    plataformas de etiquetado de imágenes populares como Cvat \cite{CVAT} o incluso conectar el
    sistema de versionado de datasets con herramientas como FiftyOne \cite{FiftyOne}.
    \item \textbf{Integración de ClearML con las tarjetas gráficas de la empresa:} Para 
    optimizar el uso de recursos de cómputo, se propone conectar ClearML con las 
    gráficas (GPUs) de la empresa. Esta integración permitirá gestionar y monitorear las 
    tareas de procesamiento de manera eficiente, asegurando una distribución adecuada de las 
    cargas de trabajo. ClearML facilitará la asignación dinámica de recursos, mejorando 
    la eficiencia del procesamiento y reduciendo tiempos de espera. Se podría implementar un 
    sistema de colas que divida las tareas según su prioridad y requerimientos de recursos. 
    Esto incluiría la creación de diferentes colas para tareas de alta prioridad, 
    tareas de procesamiento intensivo y tareas de mantenimiento. ClearML gestionaría estas colas, 
    asignando las tareas a las GPUs disponibles y asegurando un uso equilibrado y eficiente de los recursos.
    \item \textbf{Monitorización y análisis de rendimiento:} Implementar herramientas de monitorización 
    que proporcionen métricas detalladas sobre el uso de los recursos, el rendimiento de las 
    tareas y la eficiencia del sistema en general. Estas herramientas permitirían identificar 
    cuellos de botella, optimizar la asignación de recursos y planificar mejoras futuras 
    basadas en datos concretos. Además, se podría implementar un sistema de alertas que
    notifique sobre posibles problemas o anomalías en el rendimiento del sistema, permitiendo
    una respuesta rápida y eficaz ante situaciones críticas.
\end{itemize}