\subsection{Contribución al sistema de conocimiento}
Esta sección describe las contribuciones realizadas al sistema de conocimiento
durante los diferentes desarrollos realizados bajo el contexto del proyecto. 
Se presentan los componentes y plantillas que han sido almacenados una vez
terminado el proceso de validación y se han considerado útiles para futuros
proyectos.

\subsubsection{Plantillas}
\begin{table}[h!]
    \centering
    \begin{tabular}{| m{4cm} | m{7cm} | m{5cm} |}
      \hline
      \textbf{Nombre de la Plantilla} & \textbf{Descripción} & \textbf{Tags} \\ 
      \hline
      Plantilla de Factura & Plantilla para crear facturas detalladas & Factura, Contabilidad \\ 
      \hline
      Plantilla de Currículum & Plantilla para elaborar un CV profesional & CV, Trabajo, Profesional \\ 
      \hline
      Plantilla de Informe & Plantilla para redactar informes ejecutivos & Informe, Ejecutivo \\ 
      \hline
      Plantilla de Agenda & Plantilla para organizar reuniones y eventos & Agenda, Organización \\ 
      \hline
    \end{tabular}
    \caption{Tabla de Plantillas}
    \label{table:templates}    
\end{table}

\subsubsection{Componentes}
En esta subsección se presentan los componentes dibididos por categorías.
Además, se incluye una breve descripción de cada uno de ellos junto a su funcionalidad.

\subsubsubsection{Procesamiento}
\begin{itemize}
    \item \textbf{add\_time\_features:} Añade características temporales a un conjunto de datos
    de series temporales. Se pueden añadir características como el día de la semana, el mes, el año,
    la hora, entre otros. Es necesario que el argumento de entrada sea una serie temporal con la
    columna de fecha como índice.
    \item \textbf{add\_lags:} Añade retrasos a un conjunto de datos de series temporales. Se pueden
    añadir retrasos de cualquier longitud y se puede especificar si se desea eliminar las filas con
    valores nulos. Es necesario que el argumento de entrada sea una serie temporal con la columna de
    fecha como índice.
\end{itemize}

\subsubsubsection{Metricas}

\subsubsubsection{Visualización}
\begin{itemize}
    \item \textbf{show\_time\_series\_compare:} Permite comprar el resutaldo de dos series tempolales,
    mostrando un gráfico con ambas series. Además, se puede especificar el rango de fechas a mostrar.
    No es necesario que las series tengan la misma longitud, mientras que utilicen la columna de fecha
    como índice.
    \item \textbf{show\_feature\_importance:} Muestra un gráfico de barras con la importancia de las
    características de un modelo de aprendizaje automático. Se puede enviar como parámetro conjunto 
    de valores con sus respectias etiquetas y el propio componente se encarga de realizar el gráfico
    de manera descendente en funcion a la relevancia de las características.
    \item \textbf{show\_roc\_curve:} Muestra la curva ROC de un modelo de clasificación. Se puede
    enviar como parámetro el conjunto de valores reales y predichos y el componente se encarga de
    realizar el gráfico.
    \item \textbf{show\_time\_series\_outlier:} Este componente recive como paramentros de entrada
    una serie temporal y una lista de outliers. Devuelve un gráfico de la serie temporal con los
    outliers marcados. 
\end{itemize}

\subsubsubsection{Utils}
\begin{itemize}
    \item \textbf{seed\_all:}
\end{itemize}    
