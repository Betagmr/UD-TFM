\section{Estandarización de procesos de trabajo}
Esta sección describe los procesos de trabajo que se
han identificado y como se han estandarizado. Se describen
los tres procesos principales que se han identificado dentro
del proyecto: integración de una integrante en el flujo de trabajo,
desarrollo de un proyecto desde cero y contribución al sistema
de conocimiento.

\subsection{Integración en el flujo de trabajo}
La integración de nuevas personas a un grupo de trabajo es fundamental
dentro de una empresa. Cuando se añaden nuevos miembros, es natural que 
surja cierta fricción debido a diversos factores, como la adaptación a 
la dinámica del equipo, la comprensión de los procesos establecidos y la 
familiarización con las herramientas y tecnologías utilizadas. Esta fricción 
puede ralentizar el progreso del equipo.\medskip

Una forma efectiva de mejorar esta cuestión es mediante la automatización de 
procesos, como la instalación y configuración de herramientas y software necesarios 
para el trabajo del equipo. Al utilizar un script que realiza estas tareas de 
forma automática, se eliminan los posibles errores humanos y se agiliza el proceso 
de integración de los nuevos miembros. Además, al estandarizar la configuración, 
se garantiza que todos los miembros del equipo tengan el mismo entorno de trabajo, 
lo que facilita la colaboración y la comunicación. Otro beneficio que aporta 

\subsubsection{Herramientas de desarrollo}
\subsection{Desarrollo dentro de proyectos}
\subsubsection{Puesta en practica de la metodología}
\subsection{Contribución al sistema de conocimiento}

\pagebreak
