\subsection{Desarrollo dentro de proyectos}
Una vez que un nuevo miembro ha superado el periodo de integración,
ya está listo para comenzar a trabajar en proyectos. En esta subsección
se describen los procesos que se siguen para el desarrollo de un
proyecto desde cero, así como las directrices generales para
garantizar la calidad y la reproducibilidad del mismo. Para garantizar 
que esta metodología es funcional y de verdad aporta una mejora en la
se han realizado dos proyectos para cada una de las áreas de
trabajo de las series temporales, \textit{forecasting} y \textit{clasificación}
y \textit{detección de anomalías}. Por último, se ha hecho una plantilla
adicional de automl para utilizar una serie de componentes dentro de 
un proyecto de \textit{forecasting}.

\subsubsection{Flujo de trabajo}
El flujo de trabajo que se sigue para el desarrollo de un proyecto
es el siguiente:
\begin{enumerate}
    \item \textbf{Creación del proyecto:} para comenzar un proyecto, el
    investigador deberá revisar dentro de la documentación del equipo
    si existe alguna plantilla que se ajuste a sus necesidades y, en caso
    de que esta se adapte a lo que busca, podrá crear un nuevo proyecto
    a partir de esta plantilla. En caso de que no exista una plantilla
    que específica para ese caso de uso, se da la alternativa de utilizar
    una plantilla de otro caso de uso ya que puede ser que ciertas configuraciones
    específicas de librerías le sean de utilidad para no partir de cero.
    Si no se cumplen ninguno de los dos casos anteriores, se presenta una
    plantilla genérica que se puede utilizar para cualquier proyecto. En casos
    excepcionales, se puede crear un proyecto desde cero pero debe seguir los
    estándares de calidad definidos en su respectiva sección y ser perfectamente
    reproducible.
    \item \textbf{Control de versiones:} una vez que el proyecto ha sido creado,
    el investigador deberá subir el proyecto a un repositorio de control de versiones,
    en este caso se utiliza Git-Lab. El siguiente paso es utilizar las funciones de 
    ClearML para versionar los datasets que se requieran para el desarrollo del proyecto.
    No existe una imposición sobre como gestionar las ramas de git, cada investigador
    puede decidir como adaptarlo dependiendo del proyecto, equipo o el número de integrantes.
    Las directrices generales son la utilización de la rama \textit{main} para versiones
    estables del modelo, es decir un entrenamiento e inferencia funcional, y la rama 
    \textit{dev} para versiones en desarrollo.
    \item \textbf{Desarrollo y experimentación:} para garantizar el buen desarrollo del proyecto, se
    recomienda mantener el histórico de resultados de los experimentos que se realicen
    en el proyecto. Por defecto, el proyecto ya viene configurado para que se guarden
    los resultados de los experimentos en ClearML, a si que se recomienda eliminar 
    aquellos que hayan fallado o que sean irrelevantes con el objetivo de poder
    lanzar nuevamente esos experimentos en un futuro. Se recomienda seguir
    las indicaciones de los \textit{linters} y \textit{formatters} que se han configurado
    para garantizar la consistencia y claridad del código, aunque se puede desactivar
    para casos específicos como es el caso de errores de tipado. El uso del catálogo
    de componentes es opcional, pero es de gran utilidad para agilizar ciertas tareas
    repetitivas.
    \item \textbf{Aportación al sistema de conocimiento:} una vez que el proyecto
    ha sido finalizado, se recomienda aportar al sistema de conocimiento del equipo
    para que otros miembros puedan beneficiarse del nuevo conocimiento. Si
    no existía una plantilla específica para ese caso de uso, lo ideal es crear
    una nueva plantilla para esa especificación ya que es mucho más sencillo
    y requiere menos esfuerzo que crear un componente. En caso de que ya hayamos
    usado una plantilla, también se podrá crear otra si la solución que
    se ha encontrado es significativamente diferente a la original. Por último,
    si no se cumplen ninguno de los dos casos anteriores, y hay funcionalidades
    que se pueden reutilizar en otros proyectos sería recomendable crear un
    componente.
\end{enumerate}


% Hablar aquí de las tres aportaciones que se han hecho Forecasting, 
\subsubsection{Puesta en practica de la metodología}
