\section{Consideraciones éticas}
La implementación de nuevas tecnologías y paradigmas, como 
es el caso de MLOps, no solo transforma las prácticas técnicas y 
operativas dentro de una organización, sino que también plantea una 
serie de desafíos y consideraciones éticas que deben ser abordadas 
de manera rigurosa y consciente.\medskip

Estas tecnologías prometen mejoras significativas en eficiencia 
y productividad, pero también traen consigo preocupaciones en 
términos de equidad, responsabilidad, privacidad y seguridad. La 
adopción de estas prácticas disruptivas debe ser guiada no solo 
por criterios técnicos y económicos, sino también por principios 
éticos que aseguren un impacto positivo en todos los aspectos 
involucrados. Al considerar estos diversos enfoques éticos, se 
busca proporcionar una comprensión integral de los efectos 
potenciales de este proyecto y ofrecer recomendaciones para 
una adopción ética.

\subsection{Cuestiones éticas}
\subsubsection{Privacidad y seguridad de los datos}
Es crucial garantizar que los datos utilizados y generados por el 
sistema estén protegidos contra accesos no autorizados. Esto incluye 
tanto los datos personales de los investigadores como la información sensible 
de la empresa. La transparencia en el uso de datos es fundamental, 
los involucrados deben ser informados sobre cómo se recopilan, almacenan 
y utilizan sus datos dentro del sistema. Además, es importante considerar 
cómo se gestionan los datos al finalizar los proyectos, asegurando su correcta 
eliminación o anonimización conforme a las nuevas normativas europeas, 
como el Reglamento General de Protección de Datos (RGPD).

\subsubsection{Responsabilidad y transparencia}
Se debe especificar quiénes son los encargados de crear y mantener el sistema, 
asegurando su correcto funcionamiento y evitando fallos que puedan tener 
consecuencias negativas. Los desarrolladores deben garantizar que el sistema esté 
diseñado de manera robusta y eficiente, mientras que el equipo debe garantizar su desempeño 
y solucionar cualquier problema que surja. Además, es esencial que todos los procedimientos 
y algoritmos utilizados sean comprensibles para los miembros del equipo que requieran de su uso, 
de modo que se mantenga la confianza y la toma de decisiones bien formadas. La claridad en la 
asignación de responsabilidades ayuda a evitar la ambigüedad y asegura que cada miembro del 
equipo sepa exactamente cuáles son sus deberes y compromisos.

\subsubsection{Impacto en el empleo y las condiciones de trabajo}
La implementación debe ir acompañada de programas de formación para asegurar 
que todos los empleados puedan adaptarse a las nuevas tecnologías. La falta de formación 
puede resultar en inseguridad laboral y desmotivación entre los empleados. Además, es 
crucial integrar a todos los miembros del equipo en el proceso de adopción , 
garantizando que no se beneficie solo a unos pocos integrantes dentro del equipo, sino que 
todos los integrantes puedan aprovechar las ventajas que ofrece. Equilibrar 
la automatización con las necesidades humanas es esencial para evitar la pérdida de 
empleos y asegurar que los empleados sigan realizando tareas significativas. La inclusión 
de todo el personal en la transición tecnológica fortalece la cohesión del equipo y 
maximiza los beneficios para la organización en su conjunto.

\subsubsection{Impacto en el empleo y las condiciones de trabajo}
Es esencial respetar los derechos de autor y la propiedad intelectual 
al reutilizar conocimiento. Las contribuciones de todos los miembros 
del equipo deben ser reconocidas adecuadamente, asegurando que nadie se 
apropie indebidamente del trabajo de otros. Fomentar un entorno donde el 
conocimiento se comparta de manera justa y equitativa es crucial para la colaboración 
y el avance del proyecto. Además, se debe tener cuidado al incorporar contenido que 
no haya sido creado por los miembros del equipo y que se haya obtenido de internet. 
Es fundamental asegurarse de que cualquier material externo utilizado cuente con la 
licencia adecuada, como la licencia MIT, para evitar infracciones legales y éticas. La 
retención de información valiosa por parte de ciertos individuos puede perjudicar 
estos objetivos, por lo que la transparencia y el respeto a la propiedad intelectual son indispensables.

\subsubsection{Sostenibilidad y Responsabilidad Social} % TODO: Poner mejor esto
Se debe considerar cuidadosamente el impacto ambiental, especialmente en términos de 
consumo de energía y recursos. Es muy importante buscar soluciones sostenibles y 
eficientes que minimicen el impacto negativo en el medio ambiente. Además, es importante que 
las este sistema no solo beneficien a la empresa, sino que también contribuyan positivamente 
a la sociedad. Al integrar la sostenibilidad en cada etapa del ciclo de vida del proyecto, 
desde la conceptualización hasta la implementación y el mantenimiento, se asegura que el 
desarrollo tecnológico avance de manera responsable y con un compromiso claro hacia el bien común.


\subsection{Análisis desde diferente perspectivas éticas}
\subsubsection{Perspectiva utilitarista}
Desde la perspectiva del utilitarismo, que se centra en maximizar el bienestar 
general y minimizar los aspectos negativos, debe evaluarse en términos de sus 
beneficios y perjuicios para todos los involucrados. La implementación del sistema 
puede incrementar significativamente la eficiencia 
y productividad al automatizar tareas repetitivas y facilitar la reutilización 
de conocimientos, lo que beneficia a la empresa y a sus empleados. Sin embargo, 
para maximizar el bienestar, es crucial proporcionar formación adecuada a todos 
los empleados, asegurando que todos puedan adaptarse y beneficiarse de la nueva 
tecnología, evitando que solo un pequeño grupo se vea favorecido. Además, es 
esencial garantizar la equidad en el acceso a las herramientas y recursos, proteger 
la privacidad y seguridad de los datos conforme a normativas como el RGPD, y 
mantener un equilibrio entre la automatización y las tareas humanas significativas.

\subsubsection{Perspectiva de la ética de los principios}
Podemos identificar varios principios éticos que deben ser considerados en la
implementación de este sistema.

\begin{itemize}
    \item \textbf{Justicia:} este principio implica tratar a todas las personas 
    de manera equitativa y justa, asegurando que no haya discriminación 
    y que todos tengan las mismas oportunidades y beneficios.
    \item \textbf{Autonomía:} la autonomía se refiere al derecho de los individuos 
    a tomar decisiones informadas y libres sobre su propia vida y acciones, 
    respetando su capacidad para autodeterminarse y actuar según sus propias 
    convicciones y deseos.
    \item \textbf{Beneficencia:} la beneficencia actúa en beneficio de otros, 
    promoviendo su bienestar y previniendo o eliminando cualquier daño. Este 
    principio guía a las acciones que buscan maximizar los beneficios y contribuir 
    positivamente al bienestar de las personas.
    \item \textbf{No maleficencia:} este principio se centra en la obligación de 
    no causar daño a otros. Incluye la prevención de acciones que puedan resultar 
    en perjuicio o daño, asegurando que las decisiones y prácticas no tengan 
    consecuencias negativas para los involucrados.
    \item \textbf{Respeto a la propiedad intelectual:} Este principio se refiere al 
    reconocimiento y protección de las creaciones, ideas y trabajos de los 
    individuos. Implica garantizar que las contribuciones sean debidamente 
    acreditadas y que se respeten las leyes y normas relacionadas con el uso 
    y distribución del conocimiento y la información.
\end{itemize}

En base a estos principios, la implementación de este sistema debe garantizar 
que todos los empleados tengan acceso igualitario a la formación y a las nuevas 
oportunidades que brinda la tecnología, evitando que solo un pequeño grupo se vea 
beneficiado. Es crucial proteger la privacidad de los datos de los individuos y 
asegurar que estos tengan la información necesaria para tomar decisiones informadas 
sobre su uso. El sistema debe diseñarse de manera que maximice los beneficios, como 
la eficiencia y productividad, tanto para la empresa como para los empleados. 
Asimismo, es importante implementar medidas para minimizar cualquier daño potencial, 
incluyendo la protección de datos sensibles y la prevención de la pérdida de empleos 
mediante programas de formación y adaptación. Fomentar un entorno de trabajo donde 
las contribuciones individuales sean reconocidas y donde se utilice contenido externo 
de manera legal y ética es fundamental. Además, incorporar prácticas que reduzcan el 
impacto ambiental del sistema, promoviendo el uso eficiente de recursos y energías 
renovables, asegura una implementación ética y responsable del sistema, beneficiando 
a la organización, sus empleados y la sociedad en general.

\pagebreak