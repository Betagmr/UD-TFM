\section{Introducción}

\subsection{Motivación}

\subsection{Explicación del problema}

\subsection{Estructura del documento}
En esta sección, se presenta la estructura del documento de forma clara y organizada. Se 
brinda una visión general de cómo se han organizado los diferentes capítulos y secciones 
para abordar de manera coherente y completa todos los aspectos relevantes del proyecto. 
Además, se proporciona una breve descripción de cada capítulo, destacando su contenido 
y su contribución al conjunto de la memoria. Esta sección permite al lector tener una 
guía clara sobre cómo está estructurado el documento y qué puede esperar encontrar en cada 
sección.

\begin{itemize}
    \item \textbf{Introducción.} En este capítulo se presenta de forma breve el objetivo 
    principal del proyecto, su impacto deseado y la motivación detrás de su realización. 
    Además, se realiza una  breve descripción del problema a resolver y se enumeran de manera 
    ordenada los capítulos que componen el proyecto.
    \item \textbf{Antecedentes y justificación.} Se proporciona un estudio del estado del 
    arte y las últimas tendencias, y se justifican las antecedentes existentes durante el 
    desarrollo del proyecto.
    \item \textbf{Alcance y objetivos.} Se definen de manera detallada tanto el objetivo 
    principal como los objetivos secundarios del proyecto. También se establece el alcance 
    del proyecto, que se describe mediante una lista concisa de elementos que se encuentran 
    dentro y fuera del proyecto.
    \item \textbf{Metodología.} Se describe la metodología de trabajo utilizada durante el
    desarrollo del proyecto, así como la metodología creada para la resolución del problema.
    \item \textbf{Memoria técnica.} Se explican en detalle todos los aspectos técnicos del mismo. 
    Se incluyen la arquitectura del sistema integral, las herramientas utilizadas para el desarrollo, 
    los requisitos del sistema y las incidencias encontradas entre otros.
    \item \textbf{Proceso de desarrollo.} En este capitulo se presenta el proceso de desarrollo 
    utilizado en el proyecto. Se describe de manera detallada la metodología y las prácticas 
    empleadas durante la resolución del problema. Proporciona una visión general del enfoque 
    adoptado en el desarrollo del proyecto y cómo se aseguró la calidad y eficiencia en la 
    implementación del sistema. También se discuten posibles limitaciones de los métodos 
    y se proponen recomendaciones para investigaciones futuras.
    \item \textbf{Experimentación.} En este apartado se describe el proceso de experimentación 
    llevado a cabo en el proyecto. Se detallan los experimentos realizados, las diferentes
    representaciones del problema, los datos recopilados y los resultados obtenidos. Además, 
    se analizan e interpretan los resultados para sacar conclusiones relevantes y respaldar 
    las decisiones tomadas en el proyecto.  
    \item \textbf{Planificación y presupuesto.} Se detallan las fases y tareas del proyecto, se organizan 
    cronológicamente indicando su duración. También se incluye un esquema de descomposición 
    del trabajo y el plan de recursos humanos. Además, se incluyen los costes totales del proyecto, 
    incluyendo los materiales y los recursos humanos.
    \item \textbf{Conclusiones y trabajo a futuro.} Se presentan las reflexiones realizadas 
    tras la finalización del proyecto, así como las lecciones aprendidas y los conocimientos 
    adquiridos. Además, se presentan ideas o propuestas que podrían ser utilizadas o implementadas 
    en futuras investigaciones.
    \item \textbf{Abreviaturas, acrónimos y definiciones.} Se proporcionan explicaciones sobre 
    el significado de ciertos términos, acrónimos o abreviaturas mencionadas en la memoria y 
    que se consideran relevantes.
    \item \textbf{Bibliografía.} Se incluye una lista de referencias bibliográficas utilizadas
    durante el desarrollo de la memoria.
    \item \textbf{Anexos.} Se incluyen documentos independientes a la memoria del proyecto, 
    pero considerados lo suficientemente relevantes como para ser adjuntados en documentos separados.
    \begin{itemize}
        \item \textbf{Anexo I, Manual de usuario.} Se proporcionan las instrucciones necesarias 
        para que cualquier usuario, independientemente de su nivel de conocimiento sobre el tema 
        del proyecto, pueda poner en marcha el sistema inteligente y aprovechar todas sus funcionalidades.
        \item \textbf{Anexo II, Dimensión ética del proyecto.} Se realiza un análisis ético del proyecto 
        para garantizar que en su conjunto sea considerado éticamente aceptable y una contribución positiva 
        para la sociedad.
    \end{itemize}

\end{itemize}



\pagebreak