\section{Introducción}
El presente documento constituye la memoria del Proyecto de Fin de Máster (PFM) del Máster 
en Computación y Sistemas Inteligentes de la Universidad de Deusto, realizado en co\-la\-bo\-ra\-ción con 
Tecnalia Research and Innovation, un centro de investigación aplicada. Su propósito es documentar 
de forma rigurosa el trabajo llevado a cabo, desde la conceptualización del problema hasta la 
implementación de la solución.\medskip

En la actualidad, los sistemas basados en inteligencia artificial (IA) desempeñan un papel 
muy significativo en diversos sectores como el de la medicina o la industria. Cada vez son más las 
empresas que buscan incorporar soluciones de IA para mejorar su eficiencia y competitividad. Sin embargo, 
el desarrollo de modelos de IA es un proceso complejo que demanda con\-si\-de\-ra\-ble esfuerzo y experiencia técnica, 
así como un mantenimiento continuo y una monitorización constante para garantizar su rendimiento óptimo.\medskip

En el contexto de la investigación y desarrollo, es común centrar los esfuerzos en la búsqueda de nuevas
soluciones que aplicar a los nuevos retos que surgen dentro de la industria. Sin embargo, esta se ve gravemente 
ralentizada debido a la predominancia de tareas repetitivas y procesos manuales que consumen una gran
cantidad de tiempo, pero no aportan un valor diferencial. Además, la falta de un buen sistema de gestión
del conocimiento, conlleva por parte de las empresas a la pérdida de información generada durante el desarrollo de
proyectos previos y que podría ser reutilizada en un futuro. Todo ello sumado a la ausencia de un 
marco de trabajo común, dificulta en gran medida la cooperación, ya que cada miembro del equipo requiere 
de un tiempo adicional para adaptarse a las especificaciones de cada proyecto.\medskip

Durante los últimos años, el concepto de MLOps (Machine Learning Operations) ha ido ganando popularidad 
hasta convertirse en un elemento disruptivo en cuanto a desarrollo de modelos de IA se refiere. Este 
nuevo paradigma, que toma como base las prácticas DevOps (Development Operations), busca combinar la IA 
y el desarrollo de software moderno con el objetivo de tener un mayor control sobre el ciclo de 
vida de los modelos, permitiendo una entrega continua. Estas prácticas han demostrado 
ser muy efectivas en la industria, pero en cambio no han sido totalmente acogidas en el ámbito de la 
investigación. Esto puede deberse a multitud de factores, ya sea por la resistencia al cambio, 
la falta de comprensión de sus beneficios o la falta de recursos dedicados a su implementación.\medskip

Uno de los principales desafíos en la implementación de estas prácticas radica en las diferencias 
de conocimientos entre los miembros del equipo, lo que genera fricción. Por tanto, resulta crucial 
considerar el punto de partida del equipo y diseñar un proceso intuitivo. Este proyecto propone 
un estándar que aborda esta problemática, recopilando las mejores prácticas, tecnologías y 
procedimientos desarrollados hasta la fecha, y las adapta a un marco de trabajo común que
facilite la cooperación y la reutilización de conocimiento.

\subsection{Motivación}
Mi motivación para empezar este proyecto surge de la necesidad identificada por Tecnalia de 
explorar el ámbito del software, específicamente enfocándose en la creación de una herramienta 
que agilice el proceso de desarrollo de modelos de aprendizaje automático. Esta herramienta tiene 
como objetivo permitir que los investigadores dediquen más tiempo a la investigación en sí y 
menos a tareas repetitivas, al mismo tiempo que fomenta la cooperación y la reutilización del 
conocimiento de manera eficiente.\medskip

Desde el principio, encontré este tema sumamente interesante, especialmente considerando mi 
experiencia previa en el desarrollo de software. Percibí la oportunidad de aportar una 
solución innovadora que podría ser de gran ayuda para abordar los desafíos identificados por Tecnalia.\medskip

Además, el hecho de que la integración de buenas prácticas en el desarrollo de modelos de IA 
sea un tema en constante crecimiento, pero aún poco explorado en el ámbito de la investigación, 
me motiva aún más. Esta situación me brinda la oportunidad de realizar una contribución significativa 
que no solo beneficiará a Tecnalia, sino que también podría ayudar a otros equipos de investigación 
que se enfrenten a problemáticas similares.

\subsection{Estructura del documento}
En esta sección, se presenta la estructura del documento. Se 
brinda una visión general de cómo se han organizado los diferentes capítulos y secciones 
para abordar de manera general todos los aspectos relevantes del proyecto. 
Además, se proporciona una breve descripción de cada capítulo, destacando su contenido 
y su contribución al conjunto de la memoria.

\begin{itemize}
    \item \textbf{Introducción.} En este capítulo se presenta de forma breve el objetivo 
    principal del proyecto, su impacto deseado y la motivación detrás de su realización. 
    Además, se realiza una  breve descripción del problema a resolver y se enumeran
    los capítulos que componen el proyecto.
    \item \textbf{Antecedentes y justificación.} Se proporciona un estudio del estado del 
    arte y las últimas tendencias, y se justifican las antecedentes existentes durante el 
    desarrollo del proyecto.
    \item \textbf{Objetivos y alcance.} Se definen tanto el objetivo 
    principal como los objetivos secundarios del proyecto. También se establece el alcance 
    del proyecto, que se describe mediante una lista de elementos que se encuentran 
    dentro y fuera del proyecto.
    \item \textbf{Desarrollo del proyecto.} Se explican en detalle todos los aspectos técnicos del mismo. 
    Se incluyen la arquitectura del sistema integral, las herramientas utilizadas para el desarrollo, 
    los requisitos del sistema y las incidencias encontradas entre otros.
    \item \textbf{Planificación y presupuesto.} Se detallan las fases y tareas del proyecto de forma
    cronológica e indicando su duración. También se incluye un esquema de descomposición 
    del trabajo y el plan de recursos humanos. Además, se incluyen los costes totales del proyecto, 
    incluyendo los materiales y los recursos humanos.
    \item \textbf{Dimensión ética del proyecto.} Se realiza un análisis ético del proyecto 
    para garantizar que en su conjunto sea considerado éticamente aceptable y una contribución positiva 
    para la sociedad.
    \item \textbf{Conclusiones y trabajo a futuro.} Se presentan las reflexiones realizadas 
    tras la finalización del proyecto, así como las lecciones aprendidas y los conocimientos 
    adquiridos. Además, se presentan ideas o propuestas que podrían ser utilizadas o implementadas 
    en futuras investigaciones.
    \item \textbf{Bibliografía.} Se incluye una lista de referencias bibliográficas utilizadas
    durante el desarrollo de la memoria.

\end{itemize}



\pagebreak