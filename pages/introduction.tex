\section{Introducción}
Dentro del sector de la investigación y el desarrollo de proyectos de inteligencia artificial (IA),
es común centrar los esfuerzos en la búsqueda de nuevos métodos que aplicar a aspectos 
concretos dentro de una temática. Sin embargo, en la mayoría de los casos,
estos proyectos están repletos de tareas repetitivas y procesos manuales que consumen una gran
cantidad de tiempo y no aportan ningún tipo de valor. La falta de un buen sistema de gestión
del conocimiento, conlleva a la pérdida de información valiosa que ha sido descubierta durante el
desarrollo y que podría ser reutilizada en futuras investigaciones. Además, la ausencia de un 
marco de trabajo común, dificulta en gran medida la cooperación, ya que cada miembro del equipo requiere 
de un tiempo adicional para adaptarse a las especificaciones de cada proyecto.\medskip

Durante los últimos años, el concepto de MLOps (Machine Learning Operations) ha ido ganando popularidad 
hasta convertirse en un elemento disruptivo en cuanto a desarrollo de modelos de IA se refiere. Este 
nuevo paradigma, que toma como base las prácticas DevOps (Development Operations), busca combinar la IA 
y el desarrollo de software moderno con el objetivo de tener un mayor control sobre el ciclo de 
vida de los modelos, permitiendo una entrega continua. Estas prácticas han demostrado 
ser muy efectivas en la industria, pero en cambio no han sido totalmente acogidas en el ámbito de la 
investigación. Esto puede deberse a multitud de factores, ya sea por el cambio de mentalidad que requieren, el
desconocimiento de las ventajas que aportan estas práctica o simplemente porque no se le da la
importancia ni los recursos necesarios.\medskip

Una de los principales retos la implementación de este tipo prácticas es la fricción que se produce
entre los miembros de un equipo, ya que cada uno de ellos cuenta con unos conocimientos diferentes.
Es por ello que se hace necesario a la hora de llevarlo a la práctica, tener en cuenta la situación
de partida del equipo para hacer que el proceso sea lo más intuitivo posible. Este proyecto busca 
abordar este problema, proponiendo un estándar que haga frente a esta necesidades recopilando las 
mejores tendencias que se han ido desarrollando pero con una visión más actualizada.


\subsection{Motivación}
Mi motivación para abordar este proyecto surge de la necesidad por parte de Tecnalia de
investigar acerca de software, en concreto, la creación de una herramienta capaz de agilizar
el desarrollo de modelos de aprendizaje automático, que permita a los investigadores centrarse en 
la investigación y no en tareas repetitivas mientras de forma en la que se maximice la cooperación 
y la reutilización de conocimiento. me pareció un tema muy interesante y teniendo en cuenta mi experiencia previa en el desarrollo de software,
me pareció que podría ser de gran ayuda para dar una solución innovadora. \medskip

Además, el hecho de que el incorporar buenas practicas en el desarrollo de modelos de IA
es un tema que está en auge, pero que no ha sido muy explorado en el ámbito de la investigación,
me da la oportunidad de realizar una contribución significativa que podría ser de gran utilidad
no solo para Tecnalia, sino para cualquier equipo de investigación que se enfrente a problemáticas
similares.

\subsection{Estructura del documento}
En esta sección, se presenta la estructura del documento de forma clara y organizada. Se 
brinda una visión general de cómo se han organizado los diferentes capítulos y secciones 
para abordar de manera coherente y completa todos los aspectos relevantes del proyecto. 
Además, se proporciona una breve descripción de cada capítulo, destacando su contenido 
y su contribución al conjunto de la memoria. Esta sección permite al lector tener una 
guía clara sobre cómo está estructurado el documento y qué puede esperar encontrar en cada 
sección.

\begin{itemize}
    \item \textbf{Introducción.} En este capítulo se presenta de forma breve el objetivo 
    principal del proyecto, su impacto deseado y la motivación detrás de su realización. 
    Además, se realiza una  breve descripción del problema a resolver y se enumeran de manera 
    ordenada los capítulos que componen el proyecto.
    \item \textbf{Antecedentes y justificación.} Se proporciona un estudio del estado del 
    arte y las últimas tendencias, y se justifican las antecedentes existentes durante el 
    desarrollo del proyecto.
    \item \textbf{Alcance y objetivos.} Se definen de manera detallada tanto el objetivo 
    principal como los objetivos secundarios del proyecto. También se establece el alcance 
    del proyecto, que se describe mediante una lista concisa de elementos que se encuentran 
    dentro y fuera del proyecto.
    \item \textbf{Metodología.} Se describe la metodología de trabajo utilizada durante el
    desarrollo del proyecto, así como la metodología creada para la resolución del problema.
    \item \textbf{Memoria técnica.} Se explican en detalle todos los aspectos técnicos del mismo. 
    Se incluyen la arquitectura del sistema integral, las herramientas utilizadas para el desarrollo, 
    los requisitos del sistema y las incidencias encontradas entre otros.
    \item \textbf{Proceso de desarrollo.} En este capitulo se presenta el proceso de desarrollo 
    utilizado en el proyecto. Se describe de manera detallada la metodología y las prácticas 
    empleadas durante la resolución del problema. Proporciona una visión general del enfoque 
    adoptado en el desarrollo del proyecto y cómo se aseguró la calidad y eficiencia en la 
    implementación del sistema. También se discuten posibles limitaciones de los métodos 
    y se proponen recomendaciones para investigaciones futuras.
    \item \textbf{Experimentación.} En este apartado se describe el proceso de experimentación 
    llevado a cabo en el proyecto. Se detallan los experimentos realizados, las diferentes
    representaciones del problema, los datos recopilados y los resultados obtenidos. Además, 
    se analizan e interpretan los resultados para sacar conclusiones relevantes y respaldar 
    las decisiones tomadas en el proyecto.  
    \item \textbf{Planificación y presupuesto.} Se detallan las fases y tareas del proyecto, se organizan 
    cronológicamente indicando su duración. También se incluye un esquema de descomposición 
    del trabajo y el plan de recursos humanos. Además, se incluyen los costes totales del proyecto, 
    incluyendo los materiales y los recursos humanos.
    \item \textbf{Conclusiones y trabajo a futuro.} Se presentan las reflexiones realizadas 
    tras la finalización del proyecto, así como las lecciones aprendidas y los conocimientos 
    adquiridos. Además, se presentan ideas o propuestas que podrían ser utilizadas o implementadas 
    en futuras investigaciones.
    \item \textbf{Abreviaturas, acrónimos y definiciones.} Se proporcionan explicaciones sobre 
    el significado de ciertos términos, acrónimos o abreviaturas mencionadas en la memoria y 
    que se consideran relevantes.
    \item \textbf{Bibliografía.} Se incluye una lista de referencias bibliográficas utilizadas
    durante el desarrollo de la memoria.
    \item \textbf{Anexos.} Se incluyen documentos independientes a la memoria del proyecto, 
    pero considerados lo suficientemente relevantes como para ser adjuntados en documentos separados.
    \begin{itemize}
        \item \textbf{Anexo I, Manual de usuario.} Se proporcionan las instrucciones necesarias 
        para que cualquier usuario, independientemente de su nivel de conocimiento sobre el tema 
        del proyecto, pueda poner en marcha el sistema inteligente y aprovechar todas sus funcionalidades.
        \item \textbf{Anexo II, Dimensión ética del proyecto.} Se realiza un análisis ético del proyecto 
        para garantizar que en su conjunto sea considerado éticamente aceptable y una contribución positiva 
        para la sociedad.
    \end{itemize}

\end{itemize}



\pagebreak