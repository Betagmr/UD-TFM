\section*{Resumen}
En un mundo empresarial cada vez más dinámico y competitivo, el uso de metodologías 
y estándares modernos se ha convertido en una necesidad para las organizaciones 
que buscan mantenerse relevantes y competitivas. En este contexto, la adopción de 
metodologías ágiles, MLOps (Machine Learning Operations) y otras prácticas modernas 
se presenta como un elemento fundamental para facilitar la cooperación entre equipos, 
mejorar la calidad del producto y reducir los tiempos de desarrollo.\medskip

Las empresas que logran adaptarse y adoptar estos enfoques modernos experimentan una 
serie de beneficios significativos. En primer lugar, les permite responder de manera 
más rápida a los cambios en el entorno empresarial, lo que otorga una ventaja 
competitiva crucial. Además, la incorporación de prácticas de MLOps permite a las organizaciones 
gestionar de manera eficiente los modelos de machine learning en producción, 
garantizando su rendimiento y fiabilidad a lo largo del tiempo. Esto es especialmente 
relevante en un contexto donde el uso de la inteligencia artificial y el machine 
learning está cada vez más extendido dentro de la industria.\medskip

La implementación de estos estándares no está exenta de desafíos y dificultades. 
Cambiar la forma en que una empresa opera y se organiza puede ser un proceso complejo que 
requiere un cambio cultural significativo, así como la adopción de nuevas herramientas
y tecnologías. El objetivo de este proyecto es diseñar un stack tecnológico que permita 
a una empresa adoptar estas metodologías y estándares de manera efectiva, facilitando 
la transición y reduciendo la complejidad de la misma.

\section*{Descriptores}
MLOps, Machine Learning, CI/CD, Automatización, Metodologías ágiles

\pagebreak
\blankpage 
