%%%%%%%%%%%%%%%%%%%%%%%%%%%%% Define Article %%%%%%%%%%%%%%%%%%%%%%%%%%%%%%%%%%
\documentclass{article}


%%%%%%%%%%%%%%%%%%%%%%%%%%%%% Using Packages %%%%%%%%%%%%%%%%%%%%%%%%%%%%%%%%%%
\usepackage{geometry}
\usepackage{pgfplots}
\usepackage{lipsum}
\usepackage{mdframed}
\usepackage{amsthm}
\usepackage{bm}
\usepackage{titlesec}
\usepackage{tocloft}
\usepackage{ragged2e}
\usepackage{fancyhdr}
\usepackage{glossaries}
% \usepackage[spanish]{babel}
\usepackage[sorting=none]{biblatex}

\usepackage[hidelinks]{hyperref}
\usepackage[all]{hypcap}
\usepackage{csquotes}
\usepackage{pdfpages}
\usepackage{booktabs,multirow}
\usepackage{listings}
\usepackage{dirtree}
\usepackage{eurosym}
\usepackage{tikz}
\usepackage{indentfirst}
\usepackage{xpatch}


%%%%%%%%%%%%%%%%%%%%%%%%%% Page Setting %%%%%%%%%%%%%%%%%%%%%%%%%%%%%%%%%%%%%%%
\geometry{a4paper}
\graphicspath{{img/}}
\addbibresource{bibliography.bib}
\numberwithin{figure}{section}
\numberwithin{table}{section}
\setlength{\belowcaptionskip}{5pt} 

\titleformat{\paragraph}
{\normalfont\normalsize\bfseries}{\theparagraph}{1em}{}
\titlespacing*{\paragraph}
{0pt}{3.25ex plus 1ex minus .2ex}{1.5ex plus .2ex}

\fancyhf{}
\pagestyle{fancy}
\fancypagestyle{plain}{}
\lfoot{\thepage}
\fancyhead{}
\renewcommand{\headrulewidth}{0pt}

\lhead{PROYECTO FIN DE GRADO}

\titleformat{\section}
{\rmfamily\Large\raggedleft\uppercase}{\thesection.}{0.1cm}{}[{\titlerule[0.5pt]}]
\titleformat{\subsection}
    {\rmfamily\large\uppercase}{\thesubsection.}{0.1cm}{}
\def\check{\tikz\fill[scale=0.4](0,.35) -- (.25,0) -- (1,.7) -- (.25,.15) -- cycle;}  

%%%%%%%%%%%%%%%%%%%%%%%%%%%%%%% Plotting Settings %%%%%%%%%%%%%%%%%%%%%%%%%%%%%
\usepgfplotslibrary{colorbrewer}
\pgfplotsset{width=8cm,compat=1.9}


%%%%%%%%%%%%%%%%%%%%%%%%%%%%%%% Title & Author %%%%%%%%%%%%%%%%%%%%%%%%%%%%%%%%
\title{Diseño de un STACK Tecnológico para un equipo de IA basado en MLOPs }
\author{Asier Villar}


%%%%%%%%%%%%%%%%%%%%%%%%%%%%%%% Components %%%%%%%%%%%%%%%%%%%%%%%%%%%%%%%%%%%
\newcommand{\listequationsname}{\Large{Índice de ecuaciones}}
\newcommand{\myequations}[1]{
    \addcontentsline{equ}{myequations}{\protect\numberline{\theequation}#1}
}
\newcommand{\blankpage}{% comando pagina vuota
    \clearpage
    \null
    \thispagestyle{empty}%
    \clearpage
}

\newcommand{\equationNote}[2]{
    \begin{align} \label{#2} \ensuremath{#1} \end{align} 
    \myequations{#2} \centering \small \textit{#2} \normalsize \justify 
}

% DirTree patch
\newlength{\upBranch} % shift up the text  lines <<<<
\setlength{\upBranch}{0.7ex} % 

\newlength{\tolineSpace} % blank space bellow text  lines  <<<
\setlength{\tolineSpace}{1mm}%
\makeatletter

\xpatchcmd{\dirtree} % root
{\vbox{\@nameuse{DT@body@1}}}
{\raisebox{-\tolineSpace}{\vbox{\@nameuse{DT@body@1}}}}
{}{}    

\xpatchcmd{\dirtree} % below space
{\advance\dimen\z@ by-\@nameuse{DT@lastlevel@\the\DT@countiv}\relax}
{\advance\dimen\z@ by-\tolineSpace \advance\dimen\z@ by-\@nameuse{DT@lastlevel@\the\DT@countiv}\relax}
{}{}
    
\xpatchcmd{\dirtree}% shift up the text  lines
{\kern\DT@sep\box\z@\endgraf}
{\kern\DT@sep\raisebox{-\upBranch}{\box\z@}\endgraf}
{}{}    

\makeatother

\DTsetlength{1em}{1em}{0em}{0.4pt}{0.4pt}
\setlength{\DTbaselineskip}{16pt} %minimum size for  \large
\renewcommand{\DTstyle}{\rmfamily\large}

\newlistof{myequations}{equ}{\listequationsname}
\setlength{\cftmyequationsnumwidth}{2.3em}
\setlength{\cftmyequationsindent}{1.5em}


%%%%%%%%%%%%%%%%%%%%%%%%%%%%%%% Document %%%%%%%%%%%%%%%%%%%%%%%%%%%%%%%%%%%%%
\begin{document}
    \pagenumbering{roman}
    \includepdf[pages={1}]{frontpage.pdf}
    \includepdf[pages={1}]{frontpage.pdf}
    \section*{Resumen}
En un mundo empresarial cada vez más dinámico y competitivo, el uso de metodologías 
y estándares modernos se ha convertido en una necesidad para las organizaciones 
que buscan mantenerse relevantes y competitivas. En este contexto, la adopción de 
metodologías ágiles, MLOps (Machine Learning Operations) y otras prácticas modernas 
se presenta como un elemento fundamental para facilitar la cooperación entre equipos, 
mejorar la calidad del producto y reducir los tiempos de desarrollo.\medskip

Las empresas que logran adaptarse y adoptar estos enfoques modernos experimentan una 
serie de beneficios significativos. En primer lugar, les permite responder de manera 
más rápida a los cambios en el entorno empresarial, lo que otorga una ventaja 
competitiva crucial. Además, la incorporación de prácticas de MLOps permite a las organizaciones 
gestionar de manera eficiente los modelos de machine learning en producción, 
garantizando su rendimiento y fiabilidad a lo largo del tiempo. Esto es especialmente 
relevante en un contexto donde el uso de la inteligencia artificial y el machine 
learning está cada vez más extendido dentro de la industria.\medskip

La implementación de estos estándares no está exenta de desafíos y dificultades. 
Cambiar la forma en que una empresa opera y se organiza puede ser un proceso complejo que 
requiere un cambio cultural significativo, así como la adopción de nuevas herramientas
y tecnologías. El objetivo de este proyecto es diseñar un stack tecnológico que permita 
a una empresa adoptar estas metodologías y estándares de manera efectiva, facilitando 
la transición y reduciendo la complejidad de la misma.

\section*{Descriptores}
MLOps, Machine Learning, CI/CD, Automatización, Metodologías ágiles

\pagebreak
\blankpage 
 
    \renewcommand{\listtablename}{Índice de tablas}
\tableofcontents
\blankpage
\listoftables
\blankpage
\listoffigures
\blankpage
\pagebreak 
 

    \pagenumbering{arabic}
    \section{Introducción}

\subsection{Motivación}

\subsection{Explicación del problema}

\subsection{Estructura del documento}
En esta sección, se presenta la estructura del documento de forma clara y organizada. Se 
brinda una visión general de cómo se han organizado los diferentes capítulos y secciones 
para abordar de manera coherente y completa todos los aspectos relevantes del proyecto. 
Además, se proporciona una breve descripción de cada capítulo, destacando su contenido 
y su contribución al conjunto de la memoria. Esta sección permite al lector tener una 
guía clara sobre cómo está estructurado el documento y qué puede esperar encontrar en cada 
sección.

\begin{itemize}
    \item \textbf{Introducción.} En este capítulo se presenta de forma breve el objetivo 
    principal del proyecto, su impacto deseado y la motivación detrás de su realización. 
    Además, se realiza una  breve descripción del problema a resolver y se enumeran de manera 
    ordenada los capítulos que componen el proyecto.
    \item \textbf{Antecedentes y justificación.} Se proporciona un estudio del estado del 
    arte y las últimas tendencias, y se justifican las antecedentes existentes durante el 
    desarrollo del proyecto.
    \item \textbf{Alcance y objetivos.} Se definen de manera detallada tanto el objetivo 
    principal como los objetivos secundarios del proyecto. También se establece el alcance 
    del proyecto, que se describe mediante una lista concisa de elementos que se encuentran 
    dentro y fuera del proyecto.
    \item \textbf{Metodología.} Se describe la metodología de trabajo utilizada durante el
    desarrollo del proyecto, así como la metodología creada para la resolución del problema.
    \item \textbf{Memoria técnica.} Se explican en detalle todos los aspectos técnicos del mismo. 
    Se incluyen la arquitectura del sistema integral, las herramientas utilizadas para el desarrollo, 
    los requisitos del sistema y las incidencias encontradas entre otros.
    \item \textbf{Proceso de desarrollo.} En este capitulo se presenta el proceso de desarrollo 
    utilizado en el proyecto. Se describe de manera detallada la metodología y las prácticas 
    empleadas durante la resolución del problema. Proporciona una visión general del enfoque 
    adoptado en el desarrollo del proyecto y cómo se aseguró la calidad y eficiencia en la 
    implementación del sistema. También se discuten posibles limitaciones de los métodos 
    y se proponen recomendaciones para investigaciones futuras.
    \item \textbf{Experimentación.} En este apartado se describe el proceso de experimentación 
    llevado a cabo en el proyecto. Se detallan los experimentos realizados, las diferentes
    representaciones del problema, los datos recopilados y los resultados obtenidos. Además, 
    se analizan e interpretan los resultados para sacar conclusiones relevantes y respaldar 
    las decisiones tomadas en el proyecto.  
    \item \textbf{Planificación y presupuesto.} Se detallan las fases y tareas del proyecto, se organizan 
    cronológicamente indicando su duración. También se incluye un esquema de descomposición 
    del trabajo y el plan de recursos humanos. Además, se incluyen los costes totales del proyecto, 
    incluyendo los materiales y los recursos humanos.
    \item \textbf{Conclusiones y trabajo a futuro.} Se presentan las reflexiones realizadas 
    tras la finalización del proyecto, así como las lecciones aprendidas y los conocimientos 
    adquiridos. Además, se presentan ideas o propuestas que podrían ser utilizadas o implementadas 
    en futuras investigaciones.
    \item \textbf{Abreviaturas, acrónimos y definiciones.} Se proporcionan explicaciones sobre 
    el significado de ciertos términos, acrónimos o abreviaturas mencionadas en la memoria y 
    que se consideran relevantes.
    \item \textbf{Bibliografía.} Se incluye una lista de referencias bibliográficas utilizadas
    durante el desarrollo de la memoria.
    \item \textbf{Anexos.} Se incluyen documentos independientes a la memoria del proyecto, 
    pero considerados lo suficientemente relevantes como para ser adjuntados en documentos separados.
    \begin{itemize}
        \item \textbf{Anexo I, Manual de usuario.} Se proporcionan las instrucciones necesarias 
        para que cualquier usuario, independientemente de su nivel de conocimiento sobre el tema 
        del proyecto, pueda poner en marcha el sistema inteligente y aprovechar todas sus funcionalidades.
        \item \textbf{Anexo II, Dimensión ética del proyecto.} Se realiza un análisis ético del proyecto 
        para garantizar que en su conjunto sea considerado éticamente aceptable y una contribución positiva 
        para la sociedad.
    \end{itemize}

\end{itemize}



\pagebreak
    \section{Antecedentes y justificación}
Esta sección se centra en describir el ecosistema actual del desarrollo e investigación 
de modelos de inteligencia artificial y justificar la necesidad de la creación de un
estándar dentro de un equipo. Se presentan datos e información general sobre las 
tecnologías más relevantes utilizadas en el proyecto, dando una visión general 
de por qué se eligieron, teniendo en cuenta las últimas tendencias que están
surgiendo en el ámbito del aprendizaje automático.

\subsection{Justificación}
Cada año, más empresas y organizaciones invierten recursos significativos en 
proyectos de IA. Como se muestra en la figura \ref{fig:ai-investement},
en 2021 la inversión global superó los 270 mill millones de dólares \cite{Letzing2024-nn},
lo que supone un aumento del 40\% de la inversión con respecto al año anterior.
Este crecimiento de la inversión busca aprovechar al máximo el potencial de la IA 
y sacar provecho de su ventaja competitiva. Un ejemplo notable de este fenómeno es OpenAI, 
una empresa que ha sido valorada en más de 80 mil millones de dólares \cite{noauthor_2024-uj} con el record de crecimiento 
en el numero de usuarios más rápido de la historia \cite{Armenta2023-xt}. Todo esto
gracias a su modelo de lenguaje GPT-3, que ha demostrado cómo una aplicación de 
inteligencia artificial puede impactar significativamente en la vida de las personas.

\begin{figure}[ht]
    \centering
    \includegraphics[width=\textwidth]{ai-investement.png}
    \caption{Inversión global en IA en miles de millones \cite{Letzing2024-nn}.}
    \label{fig:ai-investement}
\end{figure}

Nos encontramos en un momento en el que la demanda de equipos cualificados 
que puedan hacer frente a los nuevos desafíos es mayor que nunca. La complejidad
de los proyectos de IA también está en aumento, ya que las aplicaciones de IA
se vuelven más sofisticadas, abarcan una gama más amplia de funciones, requieren
un mayor número de datos y los modelos se vuelven cada vez más complejos. Esta
situación a forzado a las empresas a buscar nuevas formas de gestionar sus proyectos
que han llevado a la creación de nuevas metodologías, adaptadas a las necesidades
a los nuevos retos que se presentan. Si bien es cierto que mucha información es
compartida con la comunidad, existe un gran secretismo en torno a la forma de operar
de las empresas más grandes, lo que dificulta la adopción de estas metodologías.
Podemos resaltar positivamente el caso de Meta \cite{metaia}, que es el mayor 
referente en cuanto contribución y apertura de sus desarrollos en IA.\medskip

Por todo ello, es necesario realizar contribuciones que se centren en el como 
se debería operar dentro de un equipo. Se requiere de una referencia clara sobre las 
directrices que se deben aplicar para poder implementar un estándar tecnológico y
operacional dentro de un grupo de trabajo, así como las herramientas, metodologías y buenas prácticas
a seguir para hacer un uso eficiente de los recursos y obtener resultados de calidad.
Este documento representa un estándar que se ha aplicado en base a unas necesidades
concretas y, aunque no está pensado para poder ser adoptado de forma literal por otros equipos,
se muestra el camino que se ha seguido desde la adopción de plataformas MLOps hasta la creación
de un sistema de conocimiento para la reutilización de trabajo. La justificación de este
documento es exponer el trabajo realizado y servir como referencia equipos que quieran implementar
un estándar similar o busquen inspiración para crear el suyo propio.
 
%% Antecedentes -> Contenido general
%% Estado del arte -> Contribuciones de otros equipos
%% Referencias sobre replicabilidad 

\subsection{Antecedentes}
En la sección de antecedentes, se profundiza en aquellos conceptos esenciales que son 
indispensables para entender el alcance y las contribuciones de este trabajo, así como 
para situarlo dentro de un marco conceptual adecuado. Se abordan aspectos fundamentales 
que no solo proporcionan contexto, sino que también establecen las bases teóricas y 
metodológicas sobre las cuales se construye la investigación.

\subsubsection{Diseño Atómico}
El diseño atómico es una metodología de diseño que se centra en la creación
de sistemas modulares y reutilizables. La idea principal es dividir
las diferentes funcionalidades de un sistemas en sus partes más fundamentales,
de manera que cada una de estas partes pueda ser reutilizada en diferentes
contextos. Este enfoque permite tener un mayor control sobre cada una de las
partes del sistema, facilitando su mantenimiento, documentación y reutilización.
Originalmente, el diseño atómico ha sido aplicado en el diseño de interfaces
de usuario, pero su filosofía puede ser aplicada a cualquier sistema de diseño
modular. En el contexto de este proyecto, el diseño atómico se aplicará al
diseño de un sistema de componentes para el desarrollo de modelos de aprendizaje
automático.\medskip

Dentro del diseño atómico, los componentes se dividen en cinco categorías
principales, que representan diferentes niveles de abstracción. Estas categorías
son: átomos, moléculas, organismos, plantillas y páginas. Cada una de estas
categorías representa un nivel de abstracción diferente, y se relaciona con
las demás categorías de manera jerárquica. La figura \ref{fig:atomic-design}
muestra la estructura conceptual del diseño atómico. 

\begin{figure}[ht]
    \centering
    \includegraphics[width=0.7\textwidth]{atomic-design-process.png}
    \caption{Estructura conceptual diseño atómico \cite{frost2016atomic}.}\label{fig:atomic-design}
\end{figure}

A continuación, se describen brevemente cada una de las categorías:
\begin{itemize}
    \item \textbf{Átomos:} Los átomos son los componentes más básicos de un sistema
    de diseño. Representan las funcionalidades más fundamentales, solo tienen
    una responsabilidad y no dependen de otros componentes.
    \item \textbf{Moléculas:} Las moléculas son la combinación de varios átomos
    para formar una funcionalidad más compleja. Representan la combinación de
    diferentes funcionalidades básicas para formar una funcionalidad más compleja.
    \item \textbf{Organismos:} Los organismos son la combinación de varias moléculas
    y átomos para formar una funcionalidad completa.
    \item \textbf{Plantillas:} Las plantillas son la combinación de varios Organismos
    para dar forma a un contenido.
    \item \textbf{Páginas:} Las páginas son la combinación de varias plantillas.
\end{itemize}

Esta estructura jerárquica permite que los componentes sean reutilizados en
diferentes contextos, y que cada uno de ellos pueda ser modificado de manera
independiente. Además, se facilita la documentación y el mantenimiento de los
componentes, ya que cada uno de ellos es independiente de los demás. Podemos
ver multitud de ejemplos de diseño atómico en grandes empresas y que nosotros utilizamos
a diario, como por ejemplo en la creación de sistemas de diseño Microsoft Fluent
Design o Google Material Design entre otros.\medskip

Aunque el diseño atómico se ha aplicado tradicionalmente a la creación de interfaces
de usuario, su filosofía puede ser aplicada a cualquier sistema de diseño modular.
En el contexto de este proyecto, el diseño atómico se aplicará para la creación de un
sistema de componentes en el desarrollo de modelos de aprendizaje automático.
Traeremos la filosofía del diseño atómico y la adaptaremos a nuestro contexto,
con las particularidades y necesidades que requiere el desarrollo de modelos de
aprendizaje automático. 

\subsubsection{Machine Learning Operations (MLOps)}
El desarrollo de modelos de aprendizaje automático es un proceso complejo que implica
la recopilación de datos, la creación de modelos, la evaluación de los modelos y su
puesta en producción. Cada una de estas etapas requiere de diferentes herramientas y
prácticas, y es importante que estas herramientas y prácticas estén integradas de manera
coherente para garantizar la eficacia del proceso. Los principios de MLOPs son una serie de prácticas y herramientas que se utilizan
para gestionar el ciclo de vida de los modelos de aprendizaje automático. Como se muestra en la
figura \ref{fig:mlops-workflow}, este enfoque busca aplicar las mejores tendencias dentro de la 
ingeniería de software al desarrollo de modelos, con el objetivo de mejorar la eficiencia, la 
calidad y la escalabilidad.\medskip

\begin{figure}[ht]
    \centering
    \includegraphics[width=\textwidth]{mlops-workflow.png}
    \caption{Ciclo de vida MLOps}\label{fig:mlops-workflow}
\end{figure}


Algunos de los principios clave de MLOps son:

\begin{itemize}
    \item \textbf{Automatización:} la automatización es esencial para mejorar la eficiencia del
    desarrollo de modelos y prevenir errores. Esto implica automatizar tareas como la 
    generación de datos, el despliegue de modelos, evaluación y su puesta en producción.
    \item \textbf{Colaboración y reproducibilidad:} uno de los desafíos en el Aprendizaje Automático es lograr 
    reproducir resultados de manera consistente, dada la naturaleza aleatoria de los algoritmos.
    MLOps busca, en la medida de lo posible, garantizar cierta consistencia entre los resultados
    obtenidos durante diferentes ejecuciones de un modelo para facilitar la colaboración entre
    los miembros del equipo.
    \item \textbf{Monitorización:} una vez que un modelo está en producción, es importante monitorizar
    su rendimiento para garantizar su eficacia y detectar posibles problemas. La monitorización
    permite crear alertas en caso de que el modelo no funcione correctamente y tomar medidas
    para corregirlo como por ejemplo, actualizar el modelo con nuevos datos.
    \item \textbf{Gestión de versiones:} controlar las versiones de los modelos y los datos es esencial
    para garantizar la reproducibilidad y la trazabilidad de los modelos. La gestión de versiones
    permite a los equipos comprender cómo ha evolucionado un modelo a lo largo del tiempo e incluso
    retroceder a versiones anteriores fuera necesario.
\end{itemize}

\subsection{Estado del arte}
La sección de estado del arte se centra en recopilar información sobre las tendencias
actuales dentro del ámbito del aprendizaje automático y la inteligencia artificial.
Debido a la diversidad de ramas que abarca la creación de un estándar tecnológico y
operacional, se han seleccionado aquellas que se consideran más relevantes o presentan
un mayor peso en cuanto a la toma de decisiones.

\pagebreak
    \section{Objetivos y alcance}
En esta sección se introducen los objetivos del 
proyecto, habiéndose realizado una division entre el principal y los
secundarios. Ademas de ello, se presentan los elementos que forman el alcance
y los que quedan fuera del mismo.

\subsection{Objetivos generales}
El objetivo principal del proyecto es diseñar e implementar un estándar tecnológico y 
operacional que cubra las necesidades más comunes dentro de un equipo de \textit{data science}. Se 
busca agilizar los tiempos de desarrollo y estandarizar los procesos, con el fin de
facilitar la colaboración entre investigadores y la reutilización del conocimiento. A
continuación, se detallan los objetivos específicos que guiarán el desarrollo:

\begin{itemize}
    \item \textbf{Agilizar el proceso inicial de proyectos:} Optimizar las primeras etapas 
    de los proyectos, identificando y eliminando aquellos procesos repetitivos que no aportan
    valor y que puedan retrasar su puesta en marcha. 
    \item \textbf{Facilitar la colaboración entre investigadores:} Implementar herramientas y 
    métodos que fomenten una cooperación fluida y efectiva entre los miembros del equipo de 
    investigación, con el fin de potenciar la sinergia y aprovechar al máximo el conocimiento 
    colectivo.
    \item \textbf{Definir procesos mediante buenas prácticas:} Establecer un marco de trabajo 
    basado en buenas prácticas de gestión de proyectos, con el objetivo de estandarizar los 
    procesos y garantizar su eficiencia y calidad.
    \item \textbf{Automatizar el desarrollo de modelos robustos:} Investigar y aplicar 
    técnicas que contribuyan al desarrollo automático de modelos de aprendizaje automático,
    con el fin de reducir el tiempo y el esfuerzo necesarios para obtener resultados de calidad.
    \item \textbf{Promover la reutilización del conocimiento:} Desarrollar mecanismos y herramientas 
    que faciliten la captura, organización y difusión del conocimiento generado durante el desarrollo 
    de los proyectos, con el propósito de fomentar su reutilización en futuras investigaciones y 
    actividades relacionadas.
\end{itemize}

El cumplimiento de estos objetivos se espera que no solo mejore la eficiencia y la calidad de los 
proyectos, sino que también contribuya a la creación de un entorno de trabajo más colaborativo y
enriquecedor para los miembros del equipo de investigación.

\subsection{Alcance}
En esta sección se definen los límites del proyecto, estableciendo lo que está
incluido y excluido dentro mismo. Se describirá de manera detallada las
actividades que forman parte del desarrollo final, así como aquellos elementos
que no están incluidos en el alcance del proyecto. Aunque el enfoque de este 
proyecto podría aplicarse a una amplia variedad de problemas en el ámbito del 
aprendizaje automático, en el contexto de este TFM nos centraremos en tres de 
los casos más comunes dentro del marco de las series temporales: forecasting, 
clasificación y detección de anomalías. A continuación, se detallan las 
actividades que forman parte del alcance del proyecto.

\subsubsection{Dentro del alcance}
\begin{itemize}
    \item \textbf{Integración de sistemas externos:} Se incluirá la configuración de sistemas 
    externos, como plataformas MLOPs o herramientas de visualización, con las plantillas 
    de proyectos base. Esto permitirá una integración más fluida y rápida de estos sistemas con 
    los proyectos, facilitando el flujo de datos y la visualización de resultados.
    \item \textbf{Plantillas de proyectos base:} Desarrollar plantillas para 
    los tres problemas de series temporales comentados anteriormente. Estas plantillas 
    servirán como punto de partida para proyectos específicos dentro de cada uno de estos dominios
    y definirán desde el principio una estructura y un conjunto de herramientas comunes. Además,
    se incluirán ejemplos de código y documentación que faciliten su uso y comprensión.
    \item \textbf{Componentes esenciales:} Identificar y almacenar los componentes 
    esenciales de cada proyecto, incluyendo modelos, algoritmos, métricas de evaluación y
    preprocesamiento de datos. Estos componentes se almacenarán y documentarán de forma
    que puedan ser reutilizados en futuros proyectos, facilitando la transferencia de conocimiento.
    \item \textbf{Proceso de AutoML:} Diseñar y ejecutar procesos de AutoML 
    (Machine Learning Automatizado) que demostrarán cómo se pueden combinar los conocimientos 
    adquiridos de todos los proyectos para desarrollar un sistema de aprendizaje automático 
    automatizado. Este proceso utilizará las plantillas y componentes esenciales almacenados 
    para generar modelos de forma automática.
\end{itemize}

\subsubsection{Fuera del alcance}
\begin{itemize}
    \item \textbf{Desarrollo de modelos específicos:} Aunque se incluirán ejemplos de
    modelos y algoritmos, el desarrollo de modelos específicos para problemas concretos
    no forma parte del alcance de este proyecto. Se espera que los modelos desarrollados
    sean generales y puedan ser adaptados a problemas específicos por los usuarios.
    \item \textbf{Despliegue de modelos:} El despliegue de modelos en producción no forma
    parte del alcance de este proyecto. Se espera que los modelos desarrollados puedan ser
    desplegados en sistemas de producción, pero no se incluirá en este proyecto. El enfoque
    se centrará exclusivamente en el desarrollo de los mismos.
\end{itemize}

\pagebreak
    \section{Desarrollo del Proyecto}
En esta sección se detallan los aspectos más relevantes del desarrollo del proyecto.
Se profundizará en los diferentes aspectos del diseño, la implementación y la
prueba de los sistemas y componentes desarrollados. Además, se describirán las
herramientas y tecnologías utilizadas, así como los procesos y metodologías
empleadas para el desarrollo del proyecto. A continuación, se muestra una vista general de los diferentes elementos
que conforman la estructura del marco de trabajo.

\begin{figure}[ht]
    \centering
    \includegraphics[width=\textwidth]{general-vision.png}
    \caption{Vista general del proyecto}\label{fig:general-vision}
\end{figure}

Dentro de la estructura general del proyecto, se pueden identificar tres elementos
principales: la infraestructura y herramientas, el catálogo de componentes y la
documentación del proyecto. Cada uno de estos elementos se encarga de aspectos
diferentes dentro del ecosistema del proyecto, y se relacionan entre sí para formar
un sistema completo. La idea principal es que la infraestructura sea complementaria
al desarrollo, estando al alcance de los investigadores de una forma sencilla. El catálogo
de componentes y plantillas se encarga de proporcionar una base sólida sobre la que
construir los diferentes proyectos, automatizando la creación de proyectos siguiendo 
las mejores prácticas y proporcionando una base sólida sobre la que construir. Por último,
la documentación del proyecto se encarga de proporcionar una guía clara y detallada
sobre el uso de las herramientas y componentes, así como de los procesos y metodologías
empleadas en el desarrollo del proyecto.

\subsection{Infraestructura y herramientas}
\subsubsection{Descripción general de la infraestructura}
La infraestructura y las herramientas son la base sobre la que se construirán
los diferentes proyectos de aprendizaje automático. Se encargan de proporcionar
un entorno de desarrollo e investigación eficiente, que permita a los miembros
del equipo centrarse en el desarrollo de modelos sin tener que preocuparse por
la configuración. Concretamente, se han desplegado dos plataformas 
que vienen a cubrir varias de las necesidades fundamentales de los proyectos
como son la gestión y exploración de datasets, la monitorización de experimentos o
el almacenamiento de modelos de inteligencia artificial. Además, se ha añadido un sistema de autenticación 
para garantizar la seguridad y privacidad de los datos.\medskip

Esta infraestructura se ha desplegado en un servidor interno de la empresa
utilizando contenedores de Docker. La elección de esta tecnología se debe a
que permite la creación de entornos aislados y portables, lo que facilita el
despliegue de las aplicaciones. Se ha utilizado Docker Compose para
sincronizar el despliegue de los diferentes servicios, lo que permite
realizar despliegues automatizados mediante las acciones de GitLab CI. La
figura \ref{fig:internal-server} muestra una vista general de la infraestructura
desplegada en el servidor interno de la empresa, donde se pueden observar
como se integran los diferentes servicios y sus respectivas conexiones. Además,
se puede observar que todos los servicios están interconectados mediante un
proxy inverso mediante Nginx, que se encarga de redirigir las peticiones en 
función de la URL. Esto permite que todos los servicios sean
accesibles desde el exterior a través de un único punto de entrada y que el
sistema de autenticación sea común para todos ellos.

\begin{figure}[ht]
    \centering
    \includegraphics[width=\textwidth]{internal-server.png}
    \caption{Vista general del proyecto}\label{fig:internal-server}
\end{figure}

\subsubsection{Plataformas integradas}
Previo a la elección de las plataformas integradas, se realizó una evaluación
a nivel de equipo para determinar las necesidades que se debían cubrir. Se
identificaron las siguientes funcionalidades fundamentales:

\begin{itemize}
    \item F1: Versionado y almacenamiento de dataset.
    \item F2: Monitorización de experimentos.
    \item F3: Almacenamiento de modelos de inteligencia artificial.
    \item F4: Integración de métricas en datasets y visualización de resultados.
    \item F5: Exploración de datasets.
\end{itemize}

Una vez identificadas las necesidades, se consensuó un criterio de selección
para las plataformas integradas. Este criterio es un criterio de mínimos, es
decir, se seleccionarán las plataformas que cumplan con el criterio mínimo
establecido y que, además, ofrezcan funcionalidades adicionales que puedan
ser de utilidad para el equipo. El criterio de selección se basa en los
siguientes aspectos: 

\begin{itemize}
    \item \textbf{C1 (Facilidad de uso):} Se valora muy positivamente la facilidad de uso de las
    plataformas, ya que se considera que no todo el equipo no tiene experiencia
    previa en el uso de estas herramientas.
    \item \textbf{C2 (Integración con otras herramientas):} Es fundamental que las
    plataformas integradas sean compatibles con las librerías y herramientas
    que se utilizan generalmente en proyectos (TensorFlow, PyTorch, etc.).
    \item \textbf{C3 (Poca dependencia sobre la infraestructura):} Medimos la dependencia
    sobre una plataforma como el numero de acciones que se deben realizar para
    migrar un proyecto vanilla, es decir, un proyecto que no ha sido
    desarrollado con la plataforma en mente. Y penalizando aquellas practicas
    que sean propias de la plataforma y que no sean comunes en la industria.
\end{itemize}

Con estos criterios en mente, se tuvieron en cuenta las siguientes
plataformas a la hora de realizar la evaluación: MLflow, ClearML, Kedro, ZenML, Data Version Controller (DVC),
Rath, Apache Superset. Cada una de estas plataformas tiene diferentes enfoques y 
funcionalidades, pero todas ellas cubren una o varias de las necesidades fundamentales
identificadas, por lo que se consideraron candidatas para su integración en la
infraestructura. La infraestructura final se compondrá de una o varias de estas
plataformas en función de los criterios previamente establecidos. A continuación, se 
muestra un análisis detallado de las plataformas evaluadas y las funcionalidades que ofrecen.

\begin{itemize}
    \item \textbf{MLflow:} MLflow es una plataforma MLOPs de código abierto para la gestión del ciclo de vida de
    los modelos. Ofrece una interfaz de usuario para el seguimiento de experimentos, la gestión de modelos 
    y la implementación de modelos en diferentes entornos. MLflow es compatible con la mayor parte de librerías de aprendizaje 
    automático, como TensorFlow o PyTorch. Uno de los puntos fuertes de MLflow es su gran comunidad, ya que es
    una de las plataformas más utilizadas en la industria. Sin embargo, no ofrece funcionalidades relacionadas con
    la gestión, evaluación o versionado de datasets ni con la exploración de los mismos. La dependencia sobre la
    plataforma varía en función de la tarea que se quiera realizar, pero en general, es una plataforma que no
    ata al usuario a su ella. La documentación se queda corta en cuestión de claridad y ejemplos, lo que puede
    dificultar la adopción de la plataforma por parte de los miembros del equipo.
    \item \textbf{ClearML:} ClearML al igual que MLflow, es una plataforma MLOps de código abierto que ofrece
    las mismas funcionalidades que MLflow en relación a la gestión a la gestión de experimentos, pero con la
    diferencias, que este si cuenta con funcionalidades relacionadas con la gestión, evaluación o versionado de
    datasets. ClearML también es compatible con la mayor parte de librerías populares aunque no tantas ni tan
    variadas como MLflow, pero ofrece una API que permite de la integración de estas de forma manual. La
    dependencia sobre la plataforma es mínima, ya que con pocos cambios se pueden lanzar experimentos sobre
    un código base. La documentación es clara y ofrece ejemplos en formato tanto de texto como de video, con
    proyectos sencillos y claros que permiten entender rápidamente el funcionamiento de la plataforma. El 
    principal punto débil de ClearML es que no cuenta con una gran comunidad, lo que puede dificultar a
    la hora de encontrar soluciones a ciertas problemáticas. Otro de sus puntos débiles es que por defecto
    no cuenta con un sistema de autenticación robusto, lo que te obliga a implementar uno por tu cuenta.
    \item \textbf{Data Version Controller (DVC):} DVC y DVC Studio son dos herramientas de código abierto que
    están diseñadas para manejar grandes volúmenes de datos, modelos y experimentos. DVC
    se centra en el versionado y almacenamiento de datasets, mientras que DVC Studio se centra en la
    monitorización de experimentos, visualización de resultados y almacenamiento de modelos. Además, 
    DVC Live proporciona integraciones con un número considerable de librerías de aprendizaje automático.
    La documentación está bien estructurada aunque no es tan clara como la de ClearML, pero cuenta con una
    comunidad bastante activa. En cuanto a los aspectos negativos de DVC, la principal desventaja es que
    la curva de aprendizaje es bastante pronunciada, lo que dificulta su adopción. Otro punto en contra
    es la dependencia gigantesca que tienen los proyectos que usan DVC, ya que se necesita de muchos archivos
    de configuración, integraciones manuales dentro del código y un dominio completo de los comandos de la
    herramienta para poder trabajar con ella.
    \item \textbf{Kedro:} TODO:
    \item \textbf{ZenML:} TODO:
    \item \textbf{Rath:} TODO:
    \item \textbf{Apache Superset:} TODO:
    \item \textbf{Grafana:} TODO:
\end{itemize}

Para finalizar con la evaluación, se ha realizado una tabla comparativa que muestra
las funcionalidades y criterios que cumple cada una de ellas de forma resumida.
\begin{table}[ht]
    \centering 
    \begin{tabular}{lccccccccc}  
        
        \toprule
        \multirow{2}{*}{\parbox[c]{.2\linewidth}{\centering Tecnología}} & 
        \multicolumn{5}{c}{\textbf{Funcionalidades}} && 
        \multicolumn{3}{c}{\textbf{Criterios}} \\ 
        
        \cmidrule{2-6} \cmidrule{8-10}
        & {\centering F1} & {F2} & {F3}& {F4} & {F5} && {C1} & {C2} & {C3}\\
        
        \midrule
        MLflow           & --     & \check & \check & --     & --     && \check & \check & \check \\
        ClearML          & \check & \check & \check & \check & --     && \check & \check & \check \\
        DVC              & \check & \check & \check & \check & --     && --     & \check & --     \\ 
        Kedro            & --     & --     & --     & --     & --     && --     & --     & --     \\  
        ZenML            & --     & --     & --     & --     & --     && --     & --     & --     \\ 
        Rath             & --     & --     & --     & --     & \check && \check & --     & \check \\ 
        Apache Superset  & \check & --     & --     & \check & --     && --     & --     & \check \\ 
        Grafana          & \check & --     & --     & \check & --     && --     & --     & \check \\ 
        \bottomrule
        
    \end{tabular}
    \caption{Tabla comparativa de las plataformas evaluadas}
    \label{tab:comparative-table} 
\end{table}


\subsubsection{Herramientas de desarrollo}
\subsubsection{Seguridad y priviacidad}
\subsubsection{Despligue Automatizado}
\subsection{Catalogo de componentes}
El catalogo de componentes y plantillas es una herramienta de gestión
de conocimiento que permite a los integrantes de un equipo compartir,
reutilizar y colaborar en la creación de estándares para una mayor
eficiencia en el desarrollo de modelos de IA. Los componentes son
elementos que representan pequeñas funcionalidades dentro de un
proyecto que pueden ser reutilizados de una forma sencilla. Las plantillas 
por otro lado son estructuras más complejas que agrupan varios componentes, 
configuraciones y reglas de negocio para crear un modelo dentro de una 
temática específica por cada una de ellas. A todos estos elementos los hemos
denominado bajo el nombre de STAC (Simple Tecnalia AI Components).\medskip

\subsubsection{Adaptación del diseño atómico}
La metodología de diseño atómico es una técnica que se basa en la
creación de componentes, que se pueden reutilizar en diferentes partes
de un proyecto. En el desarrollo de modelos de IA se puede adaptar
esta técnica para crear componentes que representen funcionalidades
específicas, como el preprocesamiento de datos, la selección de
características, la evaluación de modelos, entre otros.\medskip

Como hemos mencionado anteriormente, los componentes se dividen
a su vez en multitud de categorías (atómicos, moleculares, organismos, plantillas, etc.)
esta división tiene sentido dentro de su concepción original pero
en el caso de los modelos de IA, la división de los componentes se puede hacer
de una forma más sencilla, ya que en el caso de la investigación en IA, una
abstracción más sencilla puede ser más útil para los investigadores. Es por
ello que se propone una división de los componentes en tres categorías:

\begin{itemize}
    \item \textbf{Componentes atómicos:} son los componentes más sencillos
    que representan una funcionalidad específica.
    \item \textbf{Componentes compuestos:} estos componentes agrupan
    varios componentes atómicos para realizar una funcionalidad más
    compleja.
    \item \textbf{Plantillas:} estructuras de proyectos completan que buscan
    solucionar un problema específico utilizando una técnica concreta. Están
    formadas a su vez por componentes compuestos y atómicos. Además, las
    plantillas también agrupan configuraciones, reglas de negocio y
    diversas integraciones con otros sistemas.
\end{itemize}

Todos los componentes y plantillas se almacenan en un repositorio
compartido, donde los integrantes del equipo pueden colaborar en la
creación de nuevos componentes y plantillas, así como en la mejora de
los existentes.

\subsubsection{Estructura del sistema de componentes}
La arquitectura que se ha decidido implementar para el sistema de componentes
es conocida como monorepo multi-paquete. Un monorepo es una práctica de 
desarrollo de software donde todos los proyectos relacionados 
se almacenan en un único repositorio de código fuente. Esto significa que en 
lugar de tener múltiples repositorios para cada uno de los componentes, todo se 
mantienen en un único lugar. Por otro lado, que sea multi-paquete significa que
que los diferentes elementos del monorepo se organizan mediante paquetes de
software, lo que facilita su distribución de forma independiente. Este tipo de 
arquitectura cuenta con varias ventajas, entre las que se encuentran:

\begin{itemize}
    \item \textbf{Facilidad de gestión:} Tener todo en un solo lugar simplifica la gestión 
    del código, las dependencias y las versiones. No es necesario alternar entre
    diferentes repositorios para hacer cambios o resolver problemas.
    \item \textbf{Consistencia:} Todos los proyectos dentro del monorepo pueden seguir 
    las mismas convenciones, estructura de carpetas, y configuraciones, 
    lo que garantiza una mayor consistencia en el código. Esto permite que el código
    sea más fácil de mantener a largo plazo.
\end{itemize}

Uno de los principales problemas que puede surgir al utilizar un monorepo es la
complejidad que puede acarrear la organización de las diferentes carpetas, ya
que al tener todo en un solo lugar, la cantidad de archivos puede llegar
a ser muy grande y contar con un nivel de anidamiento muy profundo. Para evitar este
problema, se ha decidido tomar una estructura de carpetas basada en 
\textit{Screaming architecture}, una arquitectura de software que busca anteponer la
lógica de negocio sobre las partes técnicas del sistema. En este caso, nuestra lógica de
negocio se relaciona en torno a el problema que se busca resolver, es decir los diferentes 
conjuntos de problemas de las series temporales.\medskip

La figura \ref{fig:screaming-arch} muestra un ejemplo simplificado de la estructura
de carpetas del proyecto. Por cada uno de los diferentes problemas, se ha creado una 
carpeta principal, que a su vez contiene diferentes subcarpetas que agrupan los 
diferentes componentes dentro de cada una de las temáticas (procesamiento, modelo, métricas). 
En caso de que se necesite añadir un nuevo componente, simplemente se crea una nueva
carpeta dentro de la temática correspondiente. Por ejemplo, tomando la temática preprocesamiento de datos
dentro de clasificación, se buscaría la carpeta \textit{Data-Processing} dentro
\textit{TimeSeries-Classification} y en caso de no existir, se crearía una nueva carpeta, y dentro de ella 
se añadiría el nuevo componente. \medskip

Lo que se consigue con este enfoque es que cada uno de los diferentes problemas cuente con una 
estructura similar pero que a su vez tengan la posibilidad de incluir temáticas propia. En la
figura \ref{fig:screaming-arch} se puede ver como cada uno de los problemas cuenta con componentes
de procesamiento de datos, pero que a su vez, hay algunos de ellos que cuentan con componentes
específicos de modelos o métricas. Esto se puede deber a que cada uno de los problemas tiene
una necesidades específicas o que no se han encontrado componentes que se puedan reutilizar
en este momento.

\begin{figure}[ht]
    \dirtree{%
        .1 /components.
        .2 /TimeSeries-Forecasting.
        .3 /Data-Ingestion.
        .4 ingestion-component-1.
        .4 ingestion-component-2.
        .3 /Data-Processing.
        .4 processing-component-1.
        .3 /Models.
        .4 model-component-1.
        .2 /TimeSeries-Classification.
        .3 /Data-Processing.
        .4 processing-component-2.
        .3 /Metrics.
        .4 metric-component-1.
        .2 /TimeSeries-AnomalyDetection.
        .3 /Data-Processing.
        .4 processing-component-3.
    }
    \caption{Ejemplo de estructura de carpetas basada en Screaming Architecture.}
    \label{fig:screaming-arch}
\end{figure}

\subsubsection{Empaquetado de componentes}
La idea de los componentes es que sean fácilmente integrables en cualquier
proyecto de una forma sencilla. Para ello, se ha decidido empaquetar cada
componente de forma independiente y distribuirlos dentro de un repositorio
privado, de forma que se puedan instalar utilizando pip o cualquier otro 
gestor de paquetes como poetry.\medskip

Dentro de nuestro caso de uso, queremos tener la posibilidad de instalar
solo aquellos componentes que necesitemos en cada momento, sin tener que
descargarnos todo el contenido del repositorio. Esto es especialmente 
importante ya que las dependencias de las librerías de IA pueden llegar a
ser muy pesadas. A su vez, también se quiere que todos nuestros componentes
hereden del mismo nombre de paquete, para que se puedan utilizar de una forma
transparente a la hora de importarlos en un proyecto. Es decir, si tenemos por
ejemplo un componentes de ingesta de datos llamado \textit{get\_data} y otro de 
visualización que imprime una gráfica llamado \textit{show\_graph}, la forma de 
importarlo en un proyecto sería la siguiente:

\begin{verbatim}
    $ pip install stac-show-graph stac-get-data

    >> from stac.visualization.show_graph import show_graph
    >> from stac.data_ingestion.get_data import get_data
    >> data = get_data()
    >> show_graph(data)
\end{verbatim}

Para conseguir este efecto es necesario compendrer como funciona el empaquetado
de librerías en Python. En Python, un paquete es una carpeta que contiene por lo menos
un archivo \textit{\_\_init\_\_.py} y un \textit{pyproject.toml} o \textit{setup.py} . 
La forma en la que python se importan los paquetes es a través de un sistema de rutas, 
donde se toma la ruta hacia el paquete desde el directorio raíz del proyecto. En la
figura \ref{fig:min-package} se puede ver como es el contenido de cada componente.

\begin{figure}[ht]
    \dirtree{%
        .1 /component-name.
        .2 pyproject.toml.
        .2 /stac/component-category.
        .3 \_\_init\_\_.py.
    }
    \caption{Estructura mínima de un componente empaquetado.}
    \label{fig:min-package}
\end{figure}

Esta estructura se puede repetir para cada uno de los componentes que se quieran
añadir al repositorio. No obstante, y aprovechando que cada componente está separada
en un proyecto independiente, se pueden añadir más utilidades como tests, documentación
o ejemplos de uso. Como estamos utilizando poetry para gestionar las dependencias, y
las configuraciones globales del proyecto, también se van a añadir sus archivos correspondientes. 

\begin{figure}[ht]
    \dirtree{%
        .1 /component-name.
        .2 README.md.
        .2 pyproject.toml.
        .2 poetry.lock.
        .2 Makefile.
        .2 /tests.
        .2 /stac/component-category.
        .3 \_\_init\_\_.py.
    }
    \caption{Estructura final de un componente empaquetado.}
    \label{fig:real-package}
\end{figure}

Este empaquetado puede parecer tedioso si se tiene que hacer manualmente, pero en
futuras secciones veremos como se puede automatizar este proceso para que sea lo más
sencillo posible. Es muy importante que la estructura sea la descrita para optimizar
la experiencia de desarrollo de los integrantes del equipo.

\subsubsection{Sistema de plantillas}
La herramienta que hemos utilizado para la creación de plantillas es Cookiecutter.
Cookiecutter es una utilidad para la generación de proyectos que te permite inicializar
un proyecto a partir de plantillas predefinidas. Funciona siguiendo un principio 
básico: en lugar de empezar un nuevo proyecto desde cero y tener que configurar 
todo manualmente, puedes usar una plantilla predefinida que ya incluya la estructura 
de directorios, archivos básicos, configuraciones iniciales, y cualquier otro 
componente necesario para tu proyecto.\medskip

Esto es especialmente útil en entornos donde necesitas crear múltiples proyectos 
con una estructura similar donde puede haber proyectos con requisitos comunes, 
como la configuración de un marco de trabajo específico, estructura de directorios 
estándar, o incluso configuraciones de pruebas y documentación.\medskip

\subsubsection{Integración continua y despliegue continuo}

\subsection{Documentación del proyecto}
La documentación juega un papel crucial en este proyecto, ya 
que proporciona una guía detallada sobre cómo utilizar los distintos 
componentes y plantillas disponibles. En este contexto, la adopción 
de Astro como herramienta para la generación de documentación
automática se debe a su facilidad de uso, ya que permite crear
paginas de forma sencilla utilizando Markdown, lo que facilita la
creación de documentos por los distintos integrantes del equipo.
Ya existen proyectos que han utilizado Astro para la generación
de documentación, como Microsoft con su proyecto Fluent UI que
utiliza Astro.

Concretamente, el proyecto usa Starlight, una plantilla de Astro que
proporciona una estructura de documentación predefinida, lo que
facilita la creación de documentación de forma rápida y sencilla.
Además, Starlight incluye un sistema de búsqueda y organización
de documentación, lo que facilita la navegación y búsqueda de
información en la documentación. Otra de sus ventajas es la
facilidad de customización, ya que permite modificar la apariencia
utilizando CSS o Tailwind e incluso incorporar funcionalidades
propias mediante JavaScript.

\subsubsection{Guías y manuales}
Para facilitar la integración de sistemas complejos como el aquí
propuesto, es necesario proporcionar guías que expliquen
de forma detallada cómo se debe interactuar con los diferentes
procesos. En este sentido, la documentación del proyecto incluye
un apartado de guías y manuales que proporciona esa información
de forma detallada. Además, se incluyen ejemplos de uso y casos
prácticos que ayudan a entender cómo se deben utilizar los distintos
elementos de forma interactiva y dinámica.

En este momento, la documentación incluye las siguientes guías:
\begin{itemize}
    \item Guía de introducción: proporciona una visión general del
    proyecto y que partes lo componen.
    \item Guía de instalación: explica cómo instalar las diferentes 
    herramientas en local y conseguir acceso a los servicios. 
    \item Guía de plantillas y componentes: proporciona información detallada sobre cómo
    utilizar los distintos componentes y plantillas disponibles.
    \item Guía de ClearMl: introducción a la plataforma MLOps ClearMl y cómo
    utilizarla en el proyecto.
    \item Guía de contribución: proporciona información sobre cómo
    contribuir al proyecto y cómo colaborar con otros miembros del
    equipo.
\end{itemize}

En el futuro, se espera ampliar la documentación con nuevas guías
que no solo expliquen cómo utilizar los distintos elementos sino que
también se centren en aspectos enfocados a la IA y el aprendizaje
automático, como por ejemplo guías sobre cómo solucionar problemas
relacionados con el día a día de la empresa.

\subsubsection{Construcción automática de documentación}


\subsubsection{Documentación de componentes y plantillas}

\subsubsection{Sistema de búsqueda y organización de documentación}



\pagebreak
    \section{Estandarización de procesos de trabajo}
Esta sección describe los procesos de trabajo que se
han identificado y como se han estandarizado. Se describen
los tres procesos principales que se han identificado dentro
del proyecto: integración de una integrante en el flujo de trabajo,
desarrollo de un proyecto desde cero y contribución al sistema
de conocimiento.

\subsection{Integración en el flujo de trabajo}
La integración de nuevas personas a un grupo de trabajo es fundamental
dentro de una empresa. Cuando se añaden nuevos miembros, es natural que 
surja cierta fricción debido a diversos factores, como la adaptación a 
la dinámica del equipo, la comprensión de los procesos establecidos y la 
familiarización con las herramientas y tecnologías utilizadas. Esta fricción 
puede ralentizar el progreso del equipo.\medskip

Una forma efectiva de mejorar esta cuestión es mediante la automatización de 
procesos, como la instalación y configuración de herramientas y software necesarios 
para el trabajo del equipo. Al utilizar un script que realiza estas tareas de 
forma automática, se eliminan los posibles errores humanos y se agiliza el proceso 
de integración de los nuevos miembros. Además, al estandarizar la configuración, 
se garantiza que todos los miembros del equipo tengan el mismo entorno de trabajo, 
lo que facilita la colaboración y la comunicación. Otro beneficio que aporta 

\subsubsection{Herramientas de desarrollo}
\subsection{Desarrollo dentro de proyectos}
\subsubsection{Puesta en practica de la metodología}
\subsection{Contribución al sistema de conocimiento}

\pagebreak

    \section{Conclusiones y trabajo a futuro}
En esta sección se presentan las conclusiones obtenidas tras la finalización 
del proyecto, así como las lecciones aprendidas y los conocimientos adquiridos. 
Además, se presentan ideas o propuestas que podrían ser utilizadas o implementadas 
en el futuro para mejorar o ampliar el alcance del proyecto.

\subsection{Conclusiones}
Una vez finalizado el proyecto, se presentan las siguientes conclusiones en 
base a los resultados obtenidos y analizados durante el desarrollo del mismo.
Estas representan los cuatro aspectos clave del proyecto: reducción del tiempo de desarrollo, 
mejora de la productividad, reutilización del conocimiento y colaboración. Estas conclusiones 
reflejan los beneficios tangibles que el marco tecnológico propuesto aporta a los 
equipos y sus proyectos.\medskip

La implementación ha demostrado ser eficaz en la reducción significativa del tiempo 
de de\-sarro\-llo. Al automatizar tareas repetitivas y procesos manuales, se liberan 
recursos y se acelera el progreso de los proyectos. La adopción de metodologías 
ágiles y MLOps permite responder rápidamente a los cambios del entorno empresarial, 
proporcionando una ventaja competitiva crucial en cuanto a tiempo se refiere. La 
integración de plataformas como ClearML y Rath facilita la gestión y 
monitorización de experimentos, lo que reduce el tiempo dedicado a la configuración 
y gestión manual de estos procesos.\medskip

La mejora de la productividad es uno de los beneficios más destacados de la 
implementación del marco tecnológico. La automatización de tareas y la 
estandarización de procesos permiten a los equipos centrarse en actividades 
de mayor valor añadido. Además, el uso de herramientas de colaboración y la 
integración de sistemas de gestión del conocimiento aseguran que la información 
se comparta y reutilice de manera efectiva. Esto no solo incrementa la eficiencia 
operativa, sino que también mejora la calidad del trabajo, ya que los investigadores 
pueden dedicar más tiempo a la innovación y menos a tareas repetitivas.\medskip

El diseño de un sistema de componentes reutilizables y plantillas base para 
proyectos específicos facilita la reutilización del conocimiento adquirido. 
La creación de un catálogo de componentes esenciales y la documentación exhaustiva 
del proyecto aseguran que las mejores prácticas y soluciones previas se 
puedan aplicar en futuros proyectos. Este enfoque modular y estandarizado no 
solo ahorra tiempo, sino que también mejora la coherencia y calidad de los 
desarrollos, permitiendo a los equipos construir sobre trabajos anteriores 
de manera más eficiente.\medskip

La adopción de un marco común y la integración de herramientas colaborativas 
son fundamentales para mejorar la cooperación entre los miembros del equipo. 
La estandarización de procesos y la implementación de sistemas de autenticación 
y seguridad, como Keycloak, garantizan un entorno seguro y accesible para todos 
los miembros del equipo. Además, la encuesta de satisfacción realizada a los
miembros del equipo ha sido acogida de manera positiva, lo que sugiere que
existe un interés y una aceptación generalizada del nuevo marco tecnológico.\medskip

En conclusion, en base a los resultados obtenidos y analizados durante el desarrollo
del proyecto, se puede afirmar que la implementación del marco tecnológico propuesto
ha sido exitosa. Los beneficios tangibles obtenidos son muy positivos y sugieren que
el uso de este marco tecnológico puede mejorar significativamente la eficiencia y
productividad de los equipos.

\subsection{Trabajo a futuro}
A continuación, se presentan las siguientes propuestas de mejora y ampliación
del proyecto que podrían ser implementadas en futuras iteraciones:

\begin{itemize}
    \item \textbf{Adaptación del sistema a visión por computador:} Actualmente, la infraestructura 
    está enfocada en problemas basados en series temporales, pero se podría ampliar
    su alcance para incluir problemas de visión por computador. Se podría implementar
    no solo funcionalidades de preprocesamiento de imágenes, sino también incluir
    plataformas de etiquetado de imágenes populares como Cvat \cite{CVAT} o incluso conectar el
    sistema de versionado de datasets con herramientas como FiftyOne \cite{FiftyOne}.
    \item \textbf{Integración de ClearML con las tarjetas gráficas de la empresa:} Para 
    optimizar el uso de recursos de cómputo, se propone conectar ClearML con las 
    gráficas (GPUs) de la empresa. Esta integración permitirá gestionar y monitorear las 
    tareas de procesamiento de manera eficiente, asegurando una distribución adecuada de las 
    cargas de trabajo. ClearML facilitará la asignación dinámica de recursos, mejorando 
    la eficiencia del procesamiento y reduciendo tiempos de espera. Se podría implementar un 
    sistema de colas que divida las tareas según su prioridad y requerimientos de recursos. 
    Esto incluiría la creación de diferentes colas para tareas de alta prioridad, 
    tareas de procesamiento intensivo y tareas de mantenimiento. ClearML gestionaría estas colas, 
    asignando las tareas a las GPUs disponibles y asegurando un uso equilibrado y eficiente de los recursos.
    \item \textbf{Monitorización y análisis de rendimiento:} Implementar herramientas de monitorización 
    que proporcionen métricas detalladas sobre el uso de los recursos, el rendimiento de las 
    tareas y la eficiencia del sistema en general. Estas herramientas permitirían identificar 
    cuellos de botella, optimizar la asignación de recursos y planificar mejoras futuras 
    basadas en datos concretos. Además, se podría implementar un sistema de alertas que
    notifique sobre posibles problemas o anomalías en el rendimiento del sistema, permitiendo
    una respuesta rápida y eficaz ante situaciones críticas.
\end{itemize} 
    \pagebreak 
    \printbibliography[title=Bibliografía]
    \pagebreak
\end{document}